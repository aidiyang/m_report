\chapter{Multi body system model}
\label{ch:multi_mdl}
A multibody system model describes the dynamics of the group of interconnected rigid bodies that exhibits translational and rotational displacement \citep{mur94}. In this chapter the multi body system models of an inverted double pendulum and \emph{Toro} is derived. These models are used in the filtering equations of EKF and UKF to predict the states ahead of time.
\section{Multi body system}
The humanoid robots consists of group of rigid bodies(links) connected together by set of joints. Joints are mechanical components that connects two or more rigid bodies(links) and constraint their motion. Human beings are able to produce a motion with the help of the muscular system, where as the humanoids are driven by electric, hydraulic or pneumatic actuators, that apply torques to the joints of the robot which in turn produces the motion \cite[Chapter 2]{mur94}. The motion of bodies is described by their kinematic behaviour but the dynamic behaviour results from the equilibrium of applied forces and the rate of change of momentum. The dynamics of a robot describes how the robot moves in response to the actuator forces or torques. The dynamics of a multi body system(robot) is described by the equations of motion. In general there are several approaches in deriving equations of motion of rigid body dynamic system. One such approach is by deriving Lagrange equation. Lagrangian analysis relies on the energy properties of the mechanical system to derive the equations of motion. Lagrangian \emph{L} is defined as the difference between kinetic and potential energy of the system. Lagrangian \emph{L} is $$ L(q,\dot{q}) = T(q,\dot{q}) - V(q),$$ 
where \emph{q} represents the vector of joint angles, $\dot{q}$ is the vector of angular velocities of the joints, the kinetic and potential energies are represented by \emph{T} and \emph{V}. The joint angles \emph{q} is called the generalized coordinates of the system as the joint angles uniquely determines the position of rigid bodies in the system.

The equations of motion for a rigid body system in generalized coordinates $q \in \Re^m $ is given by
\begin{align}
\label{eq:lagrange}
\frac{d}{dt} &\dfdx{L}{\dot{q}_i} - \dfdx{L}{q_i} = \Gamma_i && i=1,....,m,
\end{align}
where $\Gamma_i$ is the external force acting on the $i^{th}$ generalized coordinate \cite[Chapter 4]{mur94}.
Simplifying and rearranging the above equation will result in the general formulation of equations of motion of a multi body system. It is given by
\begin{equation}
\label{eq:dyn_mul_bdy}
M(q)\ddot{q}+C(q,\dot{q})\dot{q}+g(q) = \tau + \tau_{ext},
\end{equation}
where \emph{M(q)} is the generalized inertia matrix, $C(q,\dot{q})$ is the matrix representing Coriolis and centrifugal forces, \emph{g(q)} is the gravity vector acting on the system and $\tau$ is the actuator torque applied to the joints $\tau_{ext}$ is the external torque acting on the system.

Rearranging the Equation \ref{eq:dyn_mul_bdy} results in the forward dynamics equation of the multibody system. Forward dynamics equation is the relation between the acceleration of the multi body system in response to the forces and torques acting on the system.
\begin{equation}
    \label{eq:fwdyn}
    \ddot{q} = M(q)^{-1} (- C(q,\dot{q})\dot{q} -g(q) + \tau + \tau_{ext})
\end{equation}

Integrating the above equation once gives the velocity, integrating it twice gives the position of the multibody system. 
$$ \begin{aligned} 
    v &= \int a ,\\ 
    p &=  \iint a \text{ or } \int v , 
    \end{aligned}$$ where $a = \ddot{q}$ represents the acceleration, $p$ represents the position and $v$ represents the velocity of the multibody system.

The position and velocity are the parameters of interest for state estimation. Kalman filter algorithm requires a dynamic model to predict these parameters. To accommodate the Equation \ref{eq:fwdyn} as the prediction model, the forward dynamics equation of the multi body system is converted into system of ordinary differential equations(ODE). The ODE system is given by
\begin{equation}
    \label{eq:fwdyn_ode}
    \begin{bmatrix} \frac{dp}{dt}  \\ \frac{dv}{dt} \end{bmatrix} = 
    \begin{bmatrix} v \\ a \end{bmatrix}
\end{equation}
 The state space representation of the multi body system is setup according to Equation \ref{eq:fwdyn_ode}. The states of the system are
\begin{equation}
    x = \begin{bmatrix} q \\ \dot q \end{bmatrix}.
\end{equation}
Substituting Equation \ref{eq:fwdyn} in \ref{eq:fwdyn_ode} gives, 
\begin{equation}
    \label{eq:dyn_ss}
    \begin{bmatrix} \frac{dq}{dt} \\ \frac{d \dot q}{dt} \end{bmatrix} =
    \begin{bmatrix}
    \dot q \\
    M(q)^{-1} \left( -C(q,\dot{q})\dot{q} - g(q) + \tau \right )
    \end{bmatrix}
\end{equation}
% Inverted Double pendulum model
\section{Inverted Double Pendulum model} 
\begin{figure}[H]
	\centering
	\includegraphics[trim = 0mm 180mm 0mm 10mm, scale=0.50]{Bilder/simp_uactcase.pdf}
	\caption{A simplified humanoid robot standing on one foot}
	\label{fig:idp_scene}
\end{figure}
Inorder to check the feasibility of solving the state estimation problem with the available measurements, a preliminary subproblem is formulated. Since we are interested in estimating the underactuated degrees of freedom, we have to consider the senario where these underactuated degrees of freedom are action on the robot. Let us consider the senario where the robot is tilting around an edge of the foot by applying some external force as shown in Figure \ref{fig:idp_scene}.b. Figure \ref{fig:idp_scene} shows a simplified version of a humanoid robot standing on one leg. All the joints other that ankle joint of the robot are assumed to remain static(joints does not produce any motion) throughout the experiment. The kinematics of the robot is simplified to one joint on the ankle connecting the whole body with the foot as shown in \ref{fig:idp_scene}.a. The kinematics of \emph{Toro} is shown in Figure \ref{fig:toro_kin}. The mass block in the Figure \ref{fig:idp_scene} represents the inertia of upperbody. The inertia of leg is incorporated in the link connceting upperbody and foot. 

When robot is tilting around an edge of the foot as shown in Figure \ref{fig:idp_scene}.b, we assume there is an imaginary joint located at the edge of the foot around which the robot tilts. Figure \ref{fig:idp_scene}.b resmbles an inverted double pendulum in Figure \ref{fig:idp}.
\begin{figure}[H]
	\centering
	\includegraphics[trim= 0mm 150mm 0mm 0mm, scale=0.65]{Bilder/inv_db_pend.pdf}
	\caption[Inverted Double Pendulum]{Inverted Double Pendulum \footnotemark}
	\label{fig:idp}
\end{figure}
\footnotetext{Image source: A mathematical introduction to robotic manipulation \cite[Chapter 4, Page 164]{mur94}}
 The inverted double pendulum shown in Figure \ref{fig:idp} consists two links of length $l_1$ and $l_2$. The distance between center of mass of link 1 from the toe joint is $r_1$, $r_2$ is the distance between center of mass of link 2 from the ankle joint. In the inverted pendulum model the ankle joint is actuated but the toe joint is unactuated. The degrees of freedom of this model are toe and ankle. Toe is the underactuated degree of freedom whose motion parameters such as angle $q_1$ and angular velocity $\dot{q}_1$ needs to be estimated. The motion parameters of the inverted double pendulum are 
\begin{equation}
	 q = 
	\begin{pmatrix}
		q_{1}\\
		q_{2}
	\end{pmatrix}
	 \dot{q} = 
	\begin{pmatrix}
		\dot{q_{1}}\\
		\dot{q_{2}}
	\end{pmatrix},
\end{equation}
where \emph{q} is the vector of joint angles and $\dot{q}$ is the vector of joint angular velocities. The angle $q_1$ and angular velocity $\dot{q}_1$ are the motion parameters of toe joint. Likewise $q_2$ and $\dot{q}_2$ are the motion parameteres of ankle joint. The control input or torque applied to the ankle joint is $\tau$. 

The equation of motion of the inverted double pendulum is derived using Lagrange formulation given in Equation \ref{eq:lagrange}. The generalized coordinates of the system are $q_1$ and $q_2$, the velocities of the corresponding generalized coordinates are $\dot{q_1}$ and $\dot{q_2}$ as shown in Figure \ref{fig:idp}. The equation of motion of double pendulum written in general formulation of multi body system in Equation \ref{eq:dyn_mul_bdy} is, 
\begin{equation}
   \label{eq:dyn_idp}
	M(q_1,q_2)
	\begin{bmatrix}
		\ddot{q_{1}} \\
		\ddot{q_{2}} 
	\end{bmatrix}
	+ C(q_1,q_2,\dot{q}_1,\dot{q}_2)
    \begin{bmatrix}
		\dot{q_{1}} \\
		\dot{q_{2}} 
	\end{bmatrix}
	+ g(q_1,q_2) = 
    \begin{bmatrix} 0 \\ \tau \end{bmatrix},
\end{equation}
where, $M(q_1,q_2)$ represents the ineria matrix of the inverted double pendulum system. It is given by 
$$ M(q_1,q_2) = \begin{bmatrix}\
    \alpha+2\beta cos(q_2) & \delta + \beta cos(q_2) \\ 
    \delta + \beta cos(q_2) & \delta  \end{bmatrix},$$ where the terms $\alpha, \beta, \delta$ is given by
    $$
    \begin{aligned}
    \alpha &= \frac{1}{12} m_1 l_1^2 + \frac{1}{12} m_2 l_2^2 + m_1 r_1^2 + m_2 (r_2^2 + l_1^2),\\
    \beta &= m_2 l_1 r_2, \\
    \delta &= \frac{1}{12} m_2 l_2^2 + m_2 r_2^2.
    \end{aligned}
    $$
$C(q_1,q_2,\dot{q}_1,\dot{q}_2)$ represents the Corioli matrix of the system. It is given by 
$$C(q_1,q_2,\dot{q}_1,\dot{q}_2) = 
\begin{bmatrix}
-\beta sin(q_2) \dot{q}_2 &-\beta sin(q_2)(\dot{q}_1 + \dot{q}_2) \\
-\beta sin(q_2) \dot{q}_1 & 0
\end{bmatrix},
$$
where $\beta$ is given in the previous equation. The gravity vector is represented by $g(q_1,q_2)$. It is given by
$$g(q_1,q_2) = 
\begin{bmatrix}
(m_1 r_1 +m_2 l_1)a_g cos(q_1) + m_2 r_2 a_g cos(q_1+q_2) \\
m_2 l_2 a_g cos(q_1+q_2)
\end{bmatrix}
$$
where $a_g$ represents the acceleration due to gravity 9.81 $m/{s}^2$.

The state space representation of the inverted double pendulum system is formulated as given in Equation \ref{eq:dyn_ss}. The states of the system are $$ x = \begin{bmatrix} q_1 \\ q_2 \\ \dot q_1  \\ \dot q_2 \end{bmatrix}. $$

The measurement equation of the ODE system are formulated with the set of available measurements. The measurement equation is  
\begin{equation}
	y= \begin{bmatrix} q_2 \\ \dot q_2 \\ acc_x \\ acc_y \\ \omega \end{bmatrix},
\end{equation}
where $q_2, \dot{q}_2$ are  angle and velocity of ankle joint measured by encoders. $acc_{x},acc_{y} $ are the Cartesian accelerations along \emph{x and y axis} measured by accelerometer present in inertial measurement unit(IMU) that is located at the end of link 2 in Figure \ref{fig:idp}. The acceleration measurements can be written in terms of states of the system as,
$$ 
    \begin{aligned}
    acc_x &= -l_1 (\ddot q_1 sin(q_1) + \dot q_1^2 cos(q_1)) - ((\ddot q_1 + \ddot q_2) l_2 sin(q_1+q_2) + (\dot q_1 + \dot q_2)^2 l_2 cos(q_1+q_2)), \\
    acc_y &= l_1 (\ddot q_1 cos(q_1) - \dot q_1^2 sin(q_1)) + ((\ddot q_1 + \ddot q_2) l_2 cos(q_1+q_2) - (\dot q_1 + \dot q_2)^2 l_2 sin(q_1+q_2)), \\
    \end{aligned}
$$
where $\ddot q_1 , \ddot q_2 $ are the accelerations which is computed using forward dynamics equation. The angular velocity of the whole system $\omega$ is measured by the gyroscope present in IMU. The measurement equation of the gyroscope is $$ \omega = \dot q_1 + \dot q_2. $$ 
The estimated states $$q_1, \dot{q_1}$$ are the angle and anglular velocity of the toe or the underactuated degrees of freedom of the system.

\subsection{Kalman filtering}

% Toro model
\section{Toro}
\emph{Toro} has a complicated kinematic structure as shown in Figure \ref{fig:toro_kin}. It consists of 3 kinematic chains branching from the common base. The hip acts as the commnon base. The three chains branching from the hip are  upper body, right leg and left leg. The upper body further branches into right hand and left hand. The two arms and legs acts as the end effectors of \emph{Toro}.

\begin{figure}
\begin{center}
\includegraphics[trim= 70mm 10mm 40mm 10mm,clip,scale=0.7]{Bilder/TORO_kinematic.pdf}
\caption{Kinematic chain of \emph{Toro} \underline{Explain the joints symbols}}
\label{fig:toro_kin}
\end{center}
\end{figure}

The generalized coordinates of \emph{Toro} are represneted by a vector of joint angles $q$ and the coordinates of the base of robot $x^b$. Number of joints of a robot is the number of controllable degrees of freedom \cite[Chapter 2]{mur94}. The coordinates of the base $x^b$ are the underactuated degrees of freedom of \emph{Toro}. 

A rigid body in space has six degrees of freedom as shown in Figure \ref{fig:rbody}.
%\begin{figure}
\begin{figure}
\begin{center}
\includegraphics[trim= 30mm 100mm 10mm 120mm,scale=0.75]{Bilder/rbody_dof.pdf}
\caption[Degrees of freedom of a rigid body]{Degrees of freedom of a rigid body \footnotemark[1]}
\label{fig:rbody}
\end{center}
\end{figure}
The translational and rotational degrees of freedom of the rigid body in Figure \ref{fig:rbody} are \emph{x,y,z} and \emph{roll,pitch,yaw}. The rotational degrees of freedom \emph{roll,pitch,yaw} corresponds to rotation of the body around \textbf{X,Y,Z} axes. The base of \emph{Toro} is assumed to be a free rigid body. Since the base has all the six degrees of freedom which corresponds to degrees of freedom of rigid body floating in three dimensional space, it is called floating base. The equation of motion of a single rigid body(floating base) is given by Newton-Euler equation of motion in body coordinates \cite[Chapter 4]{mur94}. 
\footnotetext[1]{Image source:\url{http://www.cncexpo.com/Images/pitchyawroll.jpg}}
\begin{equation}
\label{eq:dyn_rig_bdy}
\begin{bmatrix}
mI_3 & \textbf{0}_3 \\ \textbf{0}_3 &\Im
\end{bmatrix}
\begin{bmatrix}
\dot{v}^b\\ \dot{\omega}^b
\end{bmatrix}
+ \begin{bmatrix}
\omega^b \times mv^b \\ 
\omega^b \times \Im w^b
\end{bmatrix}
= W^b.\\
\end{equation}
In the above equation, \emph{m} is the mass of the rigid body, $\Im$ is the moment of inertia of the rigid body. $I_3$ represents the $3 \times 3$ identity matrix and $ \textbf{0}_3$ represents the  $3 \times 3$ zero matrix. $[\dot{v}^b,\dot{\omega}^b]$ are the translational velocity and angular velocity represented in body coordinate frame. $W^b$ is the body wrench applied to the center of mass of body. The body wrench represents a force/moment pair acting on the body. It is represented by vector in $\Re^6$ as \citep{mur94}
$$ W^b = \begin{bmatrix} F \\ \tau \end{bmatrix}, $$ where $F \in \Re^3$ is the linear component and $\tau \in \Re^3$ is the rotational component of the genaralized force.

To be consistent with the multibody system dynamics formulation in Equation \ref{eq:dyn_mul_bdy} Newton Euler Equation \ref{eq:dyn_rig_bdy} is reformulated as
\begin{equation}
\label{eq:dyn_rig_bdy_sh}
M_x^b \dot V^b + C_x^b V^b+g_x^b = W^b,
\end{equation}
 where $x^b=\begin{bmatrix}p \\ \theta\end{bmatrix} \in \Re^6$ is a vector representing the position and orientation of the rigid body with respect to world coordinate frame or spatial frame. The three dimensional position vector is $p=\begin{bmatrix}p_x \\ p_y \\ p_z \end{bmatrix} \in \Re^3$ and the three dimesional orientation vector is $ \theta= \begin{bmatrix} \theta_{x} \\ \theta_{y} \\ \theta_{z} \end{bmatrix} \in \Re^3$. The orientation of the rigid body is parmeterised by Euler angles [Appendix \ref{eq:rot_full}]. 
 The body velocity $V^b$ is 
 \begin{equation}
\label{eq:body_vel}
V^b =
\begin{bmatrix}
v^b \\ \omega^b
\end{bmatrix}
= \begin{bmatrix}
R^T \dot{p} \\ (R^T \dot{R})^\vee
\end{bmatrix},
\end{equation}
where $\vee$ operator denotes the extraction of 3 dimensional vector from a skew symmetric matrix[Appendix \ref{sec:avel_trfm}]. The rotation matrix $R$ is the matrix dependent on parameterization of Euler angles [Appendix \ref{eq:rot_full}]. The acceleration $\dot{V}^b$ is the acceleration with respect to body frame. $M_x^b$ represents the inertia matrix of the rigid body in body coordinates. $C_x^b$ is the matrix representing the Coriolis and centrifugal forces acting on the system in body coordinates. $g_x^b$ is the vector representing the gravitational forces acting on the body in body coordinates. $W^b$ is the external force applied to the center of mass of the body. \footnote[2]{Inertia, Coriolis and gravity are assumed to be given with respect to the body coordinate frame for the rest of the report}.

The equations of motion of \emph{Toro} in Figure \ref{fig:toro_kin} is composed of equation of motion of the multibody system and equation of motion of single rigid body. The equation of motion of \emph{Toro} is formulated like the equation of motion for a floating base system as given in \cite{ott09}.
\begin{equation} \label{eq:dyn_biped}
\begin{bmatrix}
M_x &M_{xq} \\ M_{qx} &M_q
\end{bmatrix}
\begin{bmatrix}
\ddot{x}^b \\ \ddot{q}
\end{bmatrix}
+
\begin{bmatrix}
C_x &C_{xq} \\ C_{qx} &C_q
\end{bmatrix}
\begin{bmatrix}
\dot{x}^b \\ \dot{q}
\end{bmatrix}
+
\begin{bmatrix}
g_x \\ g_q
\end{bmatrix}
=
\begin{bmatrix}
0 \\ \tau
\end{bmatrix}
+ (J_r^b)^T W_r^b + (J_l^b)^T W_l^b,
\end{equation}
where the terms with subscript $x$ are the parameters of floating  base, the terms with subscript $q$ are the parameters of the multibody system and the terms with subscript $xq, qx$ are the coupling terms that connects the dynamics of the floating base with dynamics of the multibody system.

Equation \ref{eq:dyn_biped} can be written in a simplified form as
\begin{equation} \label{eq:dyn_sbiped}
M(y)\ddot{y} + C(y,\dot{y})\dot{y} + g(y) = \tau + (J_r^b)^T W_r^b + (J_l^b)^T W_l^b,
\end{equation}
where $y = \begin{bmatrix} x^b \\ q \end{bmatrix}$ is the vector representing the state variables of \emph{Toro}. The generalized coordinates of multibody system is $q \in \Re^{25}$ which represents the vector of joint angles. The vector of generalized velocities is $\dot{y}=\begin{bmatrix} V^{b} \\ \dot{q} \end{bmatrix} \in \Re^{31}.$ The vector of generalized accelerations is $\ddot{y}\in \Re^{31}.$  $M(y)\in \Re^{31 \times 31}$ is the inertia matrix, $C(y,\dot{y})\in \Re^{31 \times 31}$ is the matrix accounting for centrifugal and Coriolis forces. $g(y) \in \Re^{31}$ is the gravity vector. $\tau \in \Re^{31}$ is the vector of actuating torques acting on the robot, where the first six components are zero because those degrees of freedom corresponding to $x_f$ are not actuated. $J_r^b,J_l^b \in \Re^{31 \times 6}$ are the body Jacobian matrices that transforms the wrenches $W_r^b,W_l^b \in \Re^{6}$ applied in the right and left foot to generalized forces acting on the robot. From here on the super script $b$ in body Jacobian matrix and body wrenches is omitted for the sake of simplicity.

The forward dynamics equation of \emph{Toro} is
\begin{equation}
	\label{eq:motion}
	\ddot{y} = M(y)^{-1}(-C(y,\dot{y})\dot{y} - g(y) + J_r(y)^{T}W_{r} +J_l(y)^{T}W_{l} + \tau). 
\end{equation}

\subsection{State space representation:}
The state space representation of \emph{Toro} is formulated similar to the state space representation of multibody system in Equation \ref{eq:dyn_ss}.
\begin{equation}
\label{eq:newton_motion}
 \begin{bmatrix}
\dot{pos} \\ \dot{vel}
\end{bmatrix}
= \begin{bmatrix}
vel \\ acc
\end{bmatrix}
\end{equation}
acceleration(\emph{acc}) is given by the forward dynamics of the system.

The equations of motion in Equation \ref{eq:motion} should be formulated in state space form of nonlinear systems as given in \ref{eqn:nl_sys}.
\begin{comment}
General state space representation of a nonlinear system
\begin{equation}
\label{eq:dyn_nl}
	\begin{split}
	\dot{x} = f(x,u)\\
	y = h(x,u)
	\end{split}
\end{equation}
where, $x \in \Re^{n}$ is the vector representing the states of the system. $u \in \Re^{p}$ is the vector of inputs acting on the system. $y \in \Re^{m}$ is the vector of outputs of the system.\\
\end{comment}
 \emph{R} is the rotation matrix which describes the rotation of rigid body with respect to \underline{spatial frame}. It is possible to reformulate the translational velocity of the above equation in first order \underline{ODE(Ordinary Differential Equation)}. It is not straight forward to obtain the time rate of change of Euler angles $\dot{\theta}$. There exists a transformation between the $\dot{\theta}$ and angular velocity $\omega^b$.
\begin{equation}
\label{eq:transfo_angvel}
\begin{split}
\omega^b = T(\theta)\dot{\theta}
\end{split}
\end{equation}
$T(\theta)$ is the transformation matrix [Appendix \ref{sec:avel_trfm}]. 

The state space representation of the equation of motions of \emph{Toro} can be obtained by substituting Equations \ref{eq:body_vel}, \ref{eq:transfo_angvel} and \ref{eq:motion} in Equation \ref{eq:newton_motion}

\begin{equation}
\label{eq:toro}
	\dot{x} = 
	\begin{bmatrix}
	\dot{p} \\ \dot{\theta} \\ \dot{q} \\ \ddot{y}
	\end{bmatrix}
	=
	\begin{bmatrix}
	R v^b\\	
	T(\theta)^{-1} \omega_f^b \\
	\dot{q}\\
	M(y)^{-1}(-C(y,\dot{y})\dot{y} -g(y) +  J_r(y)^{T}W_{r} +J_l(y)^{T}W_{l} + \tau)	
	\end{bmatrix}
	\\
	\end{equation}	
\begin{itemize}
\item $$ y = \begin{bmatrix} p \\ \theta \\ q \end{bmatrix} = \begin{bmatrix} x_f \\ q \end{bmatrix}, \dot{y} = \begin{bmatrix} V^b \\ \dot{q}\end{bmatrix}, x = \begin{bmatrix}y \\ \dot{y}\end{bmatrix} $$  $x_f,q$ are the parameters of the floating base and joints as described in \ref{eq:motion}. $V^b$ is the body velocity as defined in \ref{eq:body_vel} and $\dot{q}$ is the velocities of the joints of the robot. \emph{x} is the vector of system states.
\item $T(\theta_{f})$ is the matrix that transforms the angular velocity $\omega_{f}^{b}$ to the time derivative of Euler angles $\dot{\theta}_{f}$. i.e $\omega_{f}^{b}=T(\theta_{f}) \dot{\theta}_{f}$. 
\item R is the rotation matrix which describes the rotation of floating base with respect to spatial frame. $ R = R_x(\theta_x) R_y(\theta_y) R_z(\theta_z)$
\end{itemize}

\section{Prediction step}
The prediction equations of EKF are given in Equation \ref{eq:ekf_predict}. For the sake of simplicity we assume the process noise acting on the model is uncorrelated. i.e The noise acting on each state is independent $$W_k = I_{n}$$. \emph{n} is the number of states. Substituting the value of $W_k$ in Equation \ref{eq:ekf_predict}
\begin{equation}
\label{eq:predict}
\begin{split}
\hat{x}_{k+1}^- &= f(\hat{x}_{k},u_{k+1},0)\\
P_{k+1}^- &= A_kP_{k}A_k^T + Q_{k}\\
\end{split}
\end{equation}
The model is discretized for the implementation of EKF. Since the time step for integration is very small $\Delta t = 1ms$ forward Euler discretization method is used to discretize the continuous time model in \ref{eq:toro}.
\begin{equation}
\label{eq:toro_dis}
	\begin{bmatrix}
	\hat{p}_{k+1}^- \\ \hat{\theta}_{k+1}^- \\ \hat{q}_{k+1}^- \\ \hat{\dot{y}}_{k+1}^-
	\end{bmatrix}
	 =   
	 \begin{bmatrix}
	 \hat{p}_k \\ \hat{\theta}_k \\ \hat{q}_k \\ \hat{\dot{y}}_{k}
	\end{bmatrix}	  
	+ \Delta t f(\hat{x}_k,u_{k+1}) \\
\end{equation}
$$ f(\hat{x}_k,u_{k+1}) = 
	\begin{bmatrix}
	R v^b_k\\	
	T(\hat{\theta}_k)^{-1} \omega_k^b\\
	\dot{q_k}\\
	M(\hat{y}_{k})^{-1}(-C(\hat{y}_{k},\hat{\dot{y}}_{k})\hat{\dot{y}}_{k} -g(\hat{y}_{k}) +  J_r(\hat{y}_{k})^{T}W_{r,k+1} +J_l(\hat{y}_{k})^{T}W_{l,k+1} + \tau_{k+1})	
	\end{bmatrix} $$
$\hat{x}(t_k) = \hat{x}(k \Delta t) = \hat{x}_k$ represents the state x at \emph{kth} sampling instant. $\hat{x}_{k+1} = \hat{x}(k \Delta t + \Delta t)$ represents the state of the system at the next sampling instant. $u_{k+1}$ is the input at sampling instant $k+1$. It is assumed that the value of the input remains constant in the interval between two sampling instant. This assumption is valid because of the zero order hold mechanism in sensor circuitry.

Equation \ref{eq:toro_dis} is used to predict the state $\hat{x}_k$ in Equation \ref{eq:predict}. For the computation of state covariance matrix $P_k^-$ in Equation \ref{eq:predict}, the \underline{Jacobian Matrix} is computed for Equation \ref{eq:toro_dis}. \underline{The Jacobian} computation of the different parts of the equation is follows,
From Equation \ref{eq:toro_dis}
\begin{enumerate}
\item $ \hat{p}_{k+1}^- = \hat{p}_k + \Delta t Rv_b$, $ \hat{p}_k = [\hat{p}_{x,k},\hat{p}_{y,k},\hat{p}_{z,k}]$
\begin{equation}
\label{eq:dpdx}
\dfdx{\hat{p}_{j,k+1}^-}{x} = \left(\dfdx{\hat{p}_{j,k+1}^-}{x_{1}}, \dfdx{\hat{p}_{j,k+1}^-}{x_{2}}, \cdots , \dfdx{\hat{p}_{j,k+1}^-}{x_{62}}\right) \in \Re^{3 \times 62}
\end{equation}
\[
 \dfdx{\hat{p}_{k+1}^-}{x_{i}} =  \left\lbrace
  \!\begin{aligned}
   &e_i & \text{if }(i=j)\\
   &\Delta t \dfdx{R}{x_i}v_b & \text{if }(3 < i \leq 6)\\
   &\textbf{0}_{3 \times 1} &\text{if}(6 < i \leq 31) \text{ or } (35 < i \leq 62) \\
   &col(R,i-31) & \text{if } 31 < i \leq 34 \\
  \end{aligned} \right.
\]
\begin{itemize}
\item \emph{j} in the subscript represents the row dimension and  \emph{i} represents the column dimension of the matrix in Equation \ref{eq:dpdx}
\item $col(X,i)$ - represents the \emph{ith} column of matrix $X$.
\item $\dfdx{R}{x_i}$ is the partial derivative of \emph{R} with respect to the state $\hat{x}_k$ (i.e euler angles [Appendix \ref{sec:rot_mat}]), $e_i$ is the unit vectors in direction of coordinate axis and  $\textbf{0}_{3 \times 1}$ is the zero vector of dimensions $3 \times 1$ [Appendix \ref{sec:symbols}].
\end{itemize}

\item $\hat{\theta}_{k+1}^- = \hat{\theta}_k + \Delta t T(\hat{\theta}_k)^{-1} \omega_k^b$
\begin{equation}
\label{eq:dthetadx}
\dfdx{\hat{\theta}_{j,k+1}^-}{x} = \left(\dfdx{\hat{\theta}_{j,k+1}^-}{x_{1}}, \dfdx{\hat{\theta}_{j,k+1}^-}{x_{2}}, \cdots , \dfdx{\hat{\theta}_{j,k+1}^-}{x_{62}}\right) \in \Re^{3 \times 62}
\end{equation}
\[
 \dfdx{\hat{\theta}_{k+1}^-}{x_{i}} = \left\lbrace
  \!\begin{aligned}
   &\textbf{0}_{3 \times 1} &\text{if}(0 < i \leq 3) \text{ or }(6 < i \leq 31) \text{ or } (35 < i \leq 62) \\
   &e_{i-3} + \Delta t \dfdx{T(\hat{\theta}_k^-)^{-1}}{x_i}\omega_k^b & \text{if}3< \text{i} \leq 6 \\
   &col(T(\hat{\theta}_k^-)^{-1},i-34) & \text{if } 31 < i \leq 34 \\
  \end{aligned} \right.
\]
\begin{itemize}
\item $\dfdx{T(\theta_k^-)^{-1}}{x_i}$ is the partial derivative of inverse of transformation matrix with respect to state [Appendix \ref{sec:avel_trfm}] 
\end{itemize}

\item $\hat{q}_{k+1}^- = \hat{q}_k + \Delta t \dot{q}_k $
\begin{equation}
\label{eq:dqdx}
\dfdx{\hat{q}_{k+1}^-}{x} = \left(\dfdx{\hat{q}_{k+1}^-}{x_{1}}, \dfdx{\hat{q}_{k+1}^-}{x_{2}}, \cdots , \dfdx{\hat{q}_{k+1}^-}{x_{62}}\right) \in \Re^{25 \times 62}
\end{equation}
\[
\dfdx{\hat{q}_{k+1}^-}{x_{i}} = 
	\begin{cases}
	l_{25,i} & \text{if } (6 < i \leq 31) \text{ or } (38 < i \leq 62) \\
	0 & \text{otherwise}   \\
	\end{cases}
\]
\begin{itemize}
\item $l_{25,i}$ is a column vector of length 25 with 1 in the \emph{ith} position and zeros in other position [Appendix \ref{sec:symbols}].
\end{itemize}
\item $\hat{\dot{y}}_{k+1}^- = \hat{\dot{y}}_{k}+ \Delta t \Lambda $
$$\Lambda = M(\hat{y}_{k})^{-1}(-C(\hat{y}_{k},\hat{\dot{y}}_{k})\hat{\dot{y}}_{k} - g(\hat{y}_{k}) + J_r(\hat{y}_{k})^{T}W_{r,k+1} +J_l(\hat{y}_{k})^{T}W_{l,k+1} + \tau_{k+1})$$
 \begin{equation}
 \label{eq:dydx}
\dfdx{\hat{\dot{y}}_{k+1}^-}{x} = \left(\dfdx{\hat{\dot{y}}_{k+1}^-}{x_{1}}, \dfdx{\hat{\dot{y}}_{k+1}^-}{x_{2}}, \cdots , \dfdx{\hat{\dot{y}}_{k+1}^-}{x_{62}}\right) \in \Re^{31 \times 62}
\end{equation}
where,
\[
\dfdx{\hat{\dot{y}}_{k+1}^-}{x_{i}} = 
\left\{ 
\!\begin{aligned}
	& \left. \!\begin{aligned}
	-M_k^{-1}\dfdx{M_k}{x_{i}}\Lambda + &M_k^{-1}\left(-\dfdx{C_k}{x_{i}}\hat{\dot{y}}_{k} -\dfdx{g_k}{x_{i}}        + \left(\dfdx{J_{r,k}^{b}}{x_{i}}\right)^{T}W_{r,k+1}\right) \\
	& + M_k^{-1}\left(\dfdx{J_{l,k}^{b}}{x_{i}}\right)^{T}W_{l,k+1}
	\end{aligned} \right\}& \text{if } 0 < i \leq 31 \\
&l_{31,(i-31)}-M_k^{-1}\dfdx{M_k}{x_{i}}\Lambda + M_k^{-1}\left(-\dfdx{C_k}{x_{i}}\hat{\dot{y}}_{k}- col(C_k,i-31)\right) & \text{if } i < 31  \\
\end{aligned}
\right.
\]
\begin{itemize}
\item $M_k= M(\hat{y}_{k}),C_k=C(\hat{y}_{k},\hat{\dot{y}}_k),J_{r,k}=J_r(\hat{y}_{k}),J_{l,k}=J_l(\hat{y}_{k})$
\end{itemize}
The system matrix $A_k$ in Equation \ref{eq:predict} is formulated by combining Equations \ref{eq:dpdx},\ref{eq:dthetadx},\ref{eq:dqdx} and \ref{eq:dydx}
\begin{equation}
\label{eq:sys_mat}
A_k = \left(
\begin{aligned}
\dfdx{\hat{p}_{k+1}^-}{x} \\
\dfdx{\hat{\theta}_{k+1}^-}{x} \\
\dfdx{\hat{q}_{k+1}^-}{x}\\
\dfdx{\hat{\dot{y}}_{k+1}^-}{x}
\end{aligned} \right)
\in \Re^{62 \times 62}
\end{equation}
Substituting Equations \ref{eq:toro_dis} and \ref{eq:sys_mat} in \ref{eq:predict} and substituting the values of process covariance $Q_k$ completes the prediction stage of EKF.

\section{Update step}
The update equation of the EKF is given in Equation \ref{eq:ekf_correct}. The measurement equation of the system is given by $$\hat{y}_{k+1} = h(\hat{x}_{k+1}^-,u_{k+1},0)$$.For the sake of simplicity let us assume the measurement of noise are independent. $$V_k = I_3$$. Substituting the assumption in \ref{eq:ekf_correct}
\begin{equation}
\label{eq:correct}
\begin{split}
K_{k+1} &= P_{k+1}^-\hat{H}_{k+1}^{T-}(\hat{H}_{k+1}^-P_{k+1}^-\hat{H}_{k+1}^{T-} + R_{k+1})^{-1}\\
\hat{x}_{k+1} &= \hat{x}_{k+1}^- + K_{k+1}(y_{k+1}-\hat{y}_{k+1})\\
P_{k+1} &= (I- K_{k+1}\hat{H}_{k+1}^-)P_{k+1}^-
\end{split}
\end{equation}
The measurements of \emph{Toro} are Cartesian accelerations($acc^b$) of the hip measured by accelerometer($IMU_{acc}$), angular velocity($\omega^b$) of the hip measured by the gyroscope($IMU_{gyro}$), joint angles($q_j$) and joint velocities($\dot{q_j}$) measured by joint encoders.
\begin{equation}
    \label{eq:y_sens}
     y_{sens} = \begin{bmatrix} acc^b \\ \omega^b \\ q_j \\ \dot{q}_j \end{bmatrix} 
\end{equation}
\begin{itemize}
    \item The simplified model of $IMU_{acc}$ is $$ acc^b= \begin{bmatrix} acc^b_x \\ acc^b_y \\ acc^b_z \end{bmatrix} =  \begin{bmatrix}\ddot{p}_x^b \\ \ddot{p}_y^b \\ \ddot{p}_z^b \end{bmatrix} - R^T \begin{bmatrix}0 \\0 \\-9.81 \end{bmatrix}$$ \emph{R} is the rotational matrix that transforms a vector in body coordinate frame to spatial frame.
    \item $\ddot{p}^b$ is computed from the forward dynamic equation \ref{eq:motion} using the predicted values of the state
    \item $\omega_{f}^{b} $ is the vector of angular rates of the hip(floating base) measured by gyroscope. The measurements are in the frame attached to the hip. 
\end{itemize}
\begin{figure}
    \begin{center}
    %trim option's parameter order: left bottom right top
    \includegraphics[trim= 20mm 150mm 20mm 50mm,scale=0.80]{Bilder/foot_topview.pdf}
    \caption{Toro feet viewed from top}
    \label{fig:biped_feet}
    \end{center}
\end{figure}
Along with the sensor measurements kinematic constraints are also considered as mesurement. When a foot is in contact with the ground the velocity of the foot is zero. Figure \ref{fig:biped_feet} shows the contact points considered for measurements. The corner points of each foot are measured with respect to spatial frame \emph{S} as shown in Figure \ref{fig:biped_feet} before starting the experiment and they are assumed to be constant throughout the experiment. The contact points of the robot does not change throughout the experiment, since we are considering the case where the robot is tilting around one edge of the foot. 
\begin{equation}
    \label{eq:y_kin}
    \begin{split}
    y_{kin} &=
    \begin{bmatrix}
    p_{contact} \\ V_{contact}
    \end{bmatrix}\\
    p_{contact} &= \begin{bmatrix}p_{RF}\\ p_{LF}\end{bmatrix}\\
     V_{contact} &= \begin{bmatrix} V_{RF}^b \\ V_{LF}^b \end{bmatrix} = \begin{bmatrix} J_r(\hat{y})^T \hat{\dot{y}} \\ J_l(\hat{y})^T\hat{\dot{y}} \end{bmatrix}
    \end{split}
\end{equation}
\begin{itemize}
\item $p_{RF},p_{LF}$ are the vectors of contact points in right foot and left foot defined with respect to spatial frame.
\item $J_r(y), J_l(y)$ are the Jacobian of right and left foot that relates the joint velocity to the velocity of right and left foot respectively. \underline{[Appendix Define Body Jacobian]}
\end{itemize}
\begin{equation}
    \begin{split}
    p_{RF} &= \begin{bmatrix} p_{A,RF}\\ p_{B,RF}\\ p_{C,RF}\\ p_{D,RF}\end{bmatrix}= \begin{bmatrix} {H}_{RF}p_{A}\\  {H}_{RF}p_{B}\\  {H}_{RF}p_{C}\\  {H}_{RF}p_{D}\end{bmatrix} \\
    p_{LF} &= \begin{bmatrix} p_{A,LF}\\ p_{B,LF}\\ p_{C,LF}\\ p_{D,LF}\end{bmatrix}= \begin{bmatrix} {H}_{LF}p_{A}\\  {H}_{LF}p_{B}\\  {H}_{LF}p_{C}\\  {H}_{LF}p_{D}\end{bmatrix} \\
    \end{split}
\end{equation}
\begin{itemize}
\item $\hat{H}_{RF},\hat{H}_{LF}$ are the homogeneous transformation matrices of the right and left foot [Appendix \ref{sec:htm}]
\item In Figure \ref{fig:biped_feet} $p_A,p_B,p_C,p_D$ are the corner points defined with respect to respective foot frame \emph{RF,LF}.
\end{itemize}
The full measurement equations of is obtained combining Equations \ref{eq:y_sens} and \ref{eq:y_kin} 
\begin{equation}
    \label{eq:y_msr}
    y_{k+1} = \begin{bmatrix} y_{sens,k+1} \\ y_{kin,k+1} \end{bmatrix}
\end{equation}
The measurement sensitivity matrix can be computed by taking the parial derivative of the measurement equation \ref{eq:y_msr} with respect to system states \emph{x}.
\begin{enumerate}
\item $\hat{acc}^b_{k+1} = \ddot{p}_{k+1}-\hat{R}^T\begin{bmatrix} 0 \\ 0 \\ -9,81 \end{bmatrix}$
\begin{equation}
    \label{eq:dacc_msrdx}
    \dfdx{\hat{acc}_{k+1}^{b-}}{x} = \dfdx{\hat{\ddot{p}}_{k+1}^{b-}}{x} + \dfdx{\hat{R}^{T-}_{k+1}}{x}\begin{bmatrix} 0 \\ 0 \\ -9,81 \end{bmatrix}  \in \Re^{3 \times 62}
\end{equation}
\begin{itemize}
    \item $\dfdx{\hat{\ddot{p}}_{k+1}^b}{x}$ is partial derivative of acceleration of body with respect to states of the system. It is computed by substituting $\hat{x}_{k+1}^-$ for $\hat{x}_k$ in Equation \ref{eq:dydx} and then subtracting  $l_{31,i-31}$ for the case \emph{i > 31}. The first three rows of the resulting matrix is the partial derivative of acceleration with respect to states.
    \item $\dfdx{\hat{R}^T_{k+1}}{x}$ is partial derivative of Rotation matrix with respect to system state.[Appendix \ref{sec:rot_mat}]
\end{itemize}

\item $\hat{\omega}^{b-}_{k+1}$
\begin{equation}
    \label{eq:dw_msrdx} 
    \dfdx{\hat{\omega}^{b-}_{k+1}}{x} = \left(\dfdx{\hat{\omega}^{b-}_{k+1}}{x_{1}}, \dfdx{\hat{\omega}^{b-}_{k+1}}{x_{2}}, \cdots , \dfdx{\hat{\omega}^{b-}_{k+1}}{x_{62}}\right) \in \Re^{3 \times 62}
\end{equation}
\[ \dfdx{\hat{\omega}^{b-}_{k+1}}{x} = 
    \begin{cases}
    l_{3,34-i} & \text{if } 34 < i \leq 37 \\
    \textbf{0}_{3,1} &\text{otherwise}
    \end{cases}
 \]  
 
\item $\hat{q}_{k+1}^-$
\begin{equation}
\label{eq:dq_msrdx}
\dfdx{\hat{q}_{k+1}^-}{x} = \left(\dfdx{\hat{q}_{k+1}^-}{x_{1}}, \dfdx{\hat{q}_{k+1}^-}{x_{2}}, \cdots , \dfdx{\hat{q}_{k+1}^-}{x_{62}}\right) \in \Re^{25 \times 62}
\end{equation}
 \[
 \dfdx{\hat{q}_{k+1}^-}{x_{i}} =
 \begin{cases}
 1 & \text{if } 7 \leq i \leq 31 \\
 0 & \text{otherwise}
 \end{cases}
 \]

\item  $\hat{\dot{q}}_{k+1}^-$
\begin{equation}
 \label{eq:ddq_msrdx}
\dfdx{\hat{\dot{q}}_{k+1}^-}{x} = \left(\dfdx{\hat{\dot{q}}_{k+1}^-}{x_{1}}, \dfdx{\hat{\dot{q}}_{k+1}^-}{x_{2}}, \cdots , \dfdx{\hat{\dot{q}}_{k+1}^-}{x_{62}}\right) \in \Re^{25 \times 62}
\end{equation}
  \[
 \dfdx{\hat{\dot{q}}_{k+1}^-}{x_{i}} =
 \begin{cases}
 1 & \text{if } 37 < i \leq 62 \\
 0 & \text{otherwise}
 \end{cases}
 \]
 \item $\hat{p}_{RF,k+1}^- = \hat{H}_{RF,k+1}^- p = \begin{bmatrix} \hat{H}_{RF,k+1}^- p_{A}\\ \hat{H}_{RF,k+1}^- p_{B}\\ \hat{H}_{RF,k+1}^- p_{C}\\ \hat{H}_{RF,k+1}^- p_{D}\end{bmatrix}$
\begin{equation}
    \label{eq:dpr_msrdx}
    \begin{split}
    &\dfdx{\hat{p}_{RF,k+1}^-}{x} = \dfdx{\hat{H}_{RF,k+1}^-}{x}p\in \Re^{12 \times 62}
 \\
     \dfdx{\hat{H}_{RF,k+1}^-}{x} = &\left( \dfdx{\hat{H}_{RF,k+1}^-}{x_1}, \dfdx{\hat{H}_{RF,k+1}^-}{x_2},\cdots, \dfdx{\hat{H}_{RF,k+1}^-}{x_{62}} \right)     
     \end{split}
\end{equation}
\begin{itemize}
     \item $\dfdx{\hat{H}_{RF,k+1}^-}{x}$ is the derivative of homogeneous transformation matrix with respect to the system states [Appendix \ref{sec:htm}].
\end{itemize}
 \item $\hat{p}_{LF,k+1}^- = \hat{H}_{LF,k+1}^- p$
\begin{equation}
    \label{eq:dpl_msrdx}
    \dfdx{\hat{p}_{LF,k+1}^-}{x} = \dfdx{\hat{H}_{LF,k+1}^-}{x}p\in \Re^{12 \times 62}
\end{equation}
\begin{itemize}
    \item $\dfdx{\hat{p}_{LF,k+1}^-}{x}$ is computed similar to $\dfdx{\hat{p}_{RF,k+1}^-}{x}$ in Equation \ref{eq:dpr_msrdx}.
\end{itemize}
\item $\hat{V}_{contact,k+1}^b = \begin{bmatrix} \hat{V}_{RF,k+1}^b \\ \hat{V}_{LF,k+1}^b \end{bmatrix} 
=\begin{bmatrix}\hat{J}_{r,k+1}^{T-} \hat{\dot{y}}_{k+1}\\ \hat{J}_{l,k+1}^{T-} \hat{\dot{y}}_{k+1}\end{bmatrix}$
\begin{equation}
    \label{eq:dv_msrdx}
     \dfdx{\hat{V}_{cnt,k+1}^-}{x} = \left( \dfdx{\hat{V}_{cnt,k+1}^-}{x_1}, \dfdx{\hat{V}_{cnt,k+1}^-}{x_2},\cdots, \dfdx{\hat{V}_{cnt,k+1}^-}{x_{62}} \right)\in \Re^{12 \times 62}
\end{equation}
\[
\dfdx{\hat{V}_{cnt,k+1}^{-}}{x_{i}} = 
	\begin{cases}
	\left(
	\begin{aligned}
	\dfdx{\hat{J}_{r,k+1}^{T-}}{x_{i}}\dot{\hat{y}}^- \\
	\dfdx{\hat{J}_{l,k+1}^{T-}}{x_{i}}\dot{\hat{y}}^- \\
	\end{aligned} \right)
	& \text{if } 1 \leq i \leq 31 \\
	\begin{pmatrix}
	col(\hat{J}_{r,k+1}^{T-},i)\\ col(\hat{J}_{l,k+1}^{T-},i)
	\end{pmatrix}
	 	& \text{if } 32 \leq i \leq 62
	\end{cases}
\]
\begin{itemize}
    \item $ \hat{V}_{cnt,k+1}^{-}= \hat{V}_{contact,k+1}^b $
\end{itemize}
\end{enumerate}
The measurement sensitivity matrix $\hat{H}_{k+1}^-$ of the system is given by Equations \ref{eq:dacc_msrdx}, \ref{eq:dw_msrdx},\ref{eq:dq_msrdx}, \ref{eq:ddq_msrdx}, \ref{eq:dpr_msrdx}, \ref{eq:dpl_msrdx} and \ref{eq:dv_msrdx}.
\begin{equation}
\hat{H}^-_{k+1} = \left(
   \begin{aligned}
   \dfdx{\hat{acc}_{k+1}^{b-}}{x} \\
   \dfdx{\hat{\omega}^{b-}_{k+1}}{x} \\
    \dfdx{\hat{q}_{k+1}^-}{x} \\
    \dfdx{\hat{\dot{q}}_{k+1}^-}{x} \\
    \dfdx{\hat{p}_{RF,k+1}^-}{x} \\
    \dfdx{\hat{p}_{LF,k+1}^-}{x} \\
	 \dfdx{\hat{V}_{cnt,k+1}^{-}}{x} 
   \end{aligned}
	 \right) \in \Re^{80 \times 62}
\end{equation}

\begin{comment}
This is a code to format lengthy equations.
\[
  \text{left hand side} =
  \begin{cases}
    \!\begin{aligned}%[b]
       & \text{a very long expression} \\
       & + \text{that continues on the next line}
    \end{aligned}           & \text{1st condition} \\%[1ex]
    \text{short expression} & \text{2nd condition}
  \end{cases}
\]
\end{comment}
\end{enumerate}
\begin{comment}
\paragraph{Observability:}
State space representation of a linear system is,
\begin{equation}
\label{eq:dyn_l}
\begin{split}
\dot{x} &= Ax + Bu\\
y &= Cx + Du.
\end{split}
\end{equation}
where, $x \in \Re^{n}$ is the vector representing the states of the system. $u \in \Re^{p}$ is the vector of inputs, $y \in \Re^{m}$ is the vector of outputs of the system. $A \in \Re^{n \times n}$ is the system matrix. $B \in \Re^{n \times p}$ is the matrix relating state and input, $C \in \Re^{m \times n}$ is the measurement matrix relating output and state, $D \in \Re^{m \times p}$ is the matrix relating input and output of the system.

Linearising a nonlinear system in Equation \ref{eqn:nl_sys} at some operating point will lead to linear system of form Eq. \ref{eq:dyn_l}. For a linear system to be observable, it should satisfy
\begin{equation}
obs =
\begin{pmatrix}
C\\ CA \\ CA^{2}\\ \vdots \\ CA^{n-1}
\end{pmatrix}
, rank(obs) =n
\end{equation}
For our system to be observable $rank(obs) = 62$.
\end{comment}
%\textbf{ Make plots from files act=datsrc/ROBOT-TILT-0807.mat est=estimates-data/est-090701.mat or *080703.mat }
%\end{document}


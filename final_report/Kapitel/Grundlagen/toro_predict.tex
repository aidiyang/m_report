\subsection{Prediction step}
\label{subsec:toro_predict}
The prediction equations of EKF given in Equation \ref{eq:ekf_predict} are as follows:
\begin{equation}
\label{eq:predict}
\begin{split}
\hat{x}_{k+1}^- &= f(\hat{x}_{k},u_{k+1},0)\\
P_{k+1}^- &= A_kP_{k}A_k^T + W_kQ_{k}W_k^T \\
\end{split}
\end{equation}
The state space model of \emph{Toro} in \ref{eq:toro} is discretized for the implementation of EKF. For smaller integration time steps $\Delta t = 1ms$ the forward Euler discretization method can be used to discretize the continuous time model. The forward Euler discretization equation is $$ x_{k+1} = x_k + \Delta t f(x),$$ where $f(x)=\dot x$ is the nonlinear function describing the system. The discretized prediction equation of \emph{Toro} is obtained by substituting $0$ for noise terms $w_\tau, w_{W_r}$ and $w_{W_l}$ in Equation \ref{eq:toro} and applying the forward Euler discretization:
\begin{equation}
\label{eq:toro_dis}
	\begin{bmatrix}
	\hat{p}_{k+1}^- \\ \hat{\theta}_{k+1}^- \\ \hat{q}_{k+1}^- \\ \hat{\dot{y}}_{k+1}^-
	\end{bmatrix}
	 =   
	 \begin{bmatrix}
	 \hat{p}_k \\ \hat{\theta}_k \\ \hat{q}_k \\ \hat{\dot{y}}_{k}
	\end{bmatrix}	  
	+ \Delta t f(\hat{x}_k,u_{k+1}) \\
\end{equation}
$$ f(\hat{x}_k,u_{k+1})\footnotemark[1] = 
	\begin{bmatrix}
	\hat R_k \hat v^b_k \\
	\hat T_k^{-1} \hat \omega_k^b  \\
	\hat{\dot{q}}_k\\
	M(\hat{y}_{k})^{-1}(-C(\hat{y}_{k},\hat{\dot{y}}_{k})\hat{\dot{y}}_{k} -g(\hat{y}_{k}) +  J_r(\hat{y}_{k})^{T}W_{r,k+1} +J_l(\hat{y}_{k})^{T}W_{l,k+1} + \tau_{k+1})	
	\end{bmatrix}  $$
where $\hat{x}(t_k) = \hat{x}(k \Delta t) = \hat{x}_k$ represents the state x at \emph{kth} sampling instant. $\hat{x}_{k+1} = \hat{x}(k \Delta t + \Delta t)$ represents the state of the system at the next sampling instant. $u_{k+1}$ is the input at sampling instant $k+1$. The inputs $\tau, W_l, W_r$ remains constant in the interval between two sampling instant because of the zero order hold mechanism in the sensor. The above equation is used to predict the state $\hat{x}_k$ ahead of time in Equation \ref{eq:predict}. 
\footnotetext[1]{$\hat R_k, \hat T_k$ are the shorthand notations for $R(\hat \theta_k)$ and $T(\hat \theta_k)$ in the equation.}

For the computation of state covariance matrix $P_k^-$ in Equation \ref{eq:predict}, the Jacobian matrix $A$ should be computed. The Jacobian matrix is the matrix of first order partial derivatives of a vector valued function \citep{wal76}.

The Jacobian matrix computation of the prediction Equations \ref{eq:toro_dis} of EKF is as follows:
\begin{enumerate}
\item The prediction equation for the position of the base is $ \hat{p}_{k+1}^- = \hat{p}_k + \Delta t \hat R_k \hat v^b_k$, where  $ \hat{p}_k = [\hat{p}_{x,k},\hat{p}_{y,k},\hat{p}_{z,k}]$ is the coordinates of position at time instant $k$. The Jacobian matrix of the equation is given by
\begin{equation}
\label{eq:dpdx}
\dfdx{\hat{p}_{j,k+1}^-}{x} = \left(\dfdx{\hat{p}_{j,k+1}^-}{x_{1}}, \dfdx{\hat{p}_{j,k+1}^-}{x_{2}}, \cdots , \dfdx{\hat{p}_{j,k+1}^-}{x_{62}}\right) \in \Re^{3 \times 62} \hspace{2cm} j=1,2,3
\end{equation}
\[
 \dfdx{\hat{p}_{k+1}^-}{x_{i}} =  \left\lbrace
  \!\begin{aligned}
   &e_i & \text{if }(i=j)\\
   &\Delta t \dfdx{\hat R_k}{x_i} \hat v^b_k & \text{if }(3 < i \leq 6)\\
   &\textbf{0}_{3 \times 1} &\text{if }(6 < i \leq 31) \text{ or } (35 < i \leq 62) \\
   &col(\hat R_k,i-31) & \text{if } 31 < i \leq 34 \\
  \end{aligned} \right.
\]where
\begin{itemize}
\item the subscript \emph{j} represents the row dimension and \emph{i} represents the column dimension of the Jacobian matrix,
\item $col(X,i)$ - represents the $i^{th}$ column of matrix $X$,
\item the partial derivative of \emph{R} with respect to the state $\hat{x}_k$ is $\dfdx{\hat R_k}{x_i}$ [Appendix \ref{sec:rot_mat}]),
\item  $e_i$ is the unit vectors in direction of coordinate axis and  $\textbf{0}_{3 \times 1}$ is the zero vector of dimensions 3 [Appendix \ref{sec:symbols}].
\end{itemize}

\item The prediction equation for the orientation of the base is $\hat{\theta}_{k+1}^- = \hat{\theta}_k + \Delta t \hat T_k^{-1} \omega_k^b$, where $\hat \theta_k = [ \theta_{x,k},\theta_{y,k},\theta_{z,k}]$ are the coordinates of orientation of the base. The Jacobian matrix of the equation is given by
\begin{equation}
\label{eq:dthetadx}
\dfdx{\hat{\theta}_{j,k+1}^-}{x} = \left(\dfdx{\hat{\theta}_{j,k+1}^-}{x_{1}}, \dfdx{\hat{\theta}_{j,k+1}^-}{x_{2}}, \cdots , \dfdx{\hat{\theta}_{j,k+1}^-}{x_{62}}\right) \in \Re^{3 \times 62}
\end{equation}
\[
 \dfdx{\hat{\theta}_{k+1}^-}{x_{i}} = \left\lbrace
  \!\begin{aligned}
   &\textbf{0}_{3 \times 1} &\text{if }(0 < i \leq 3) \text{ or }(6 < i \leq 31) \\
   &\hspace{2cm} &\text{ or } (35 < i \leq 62) \\
   &e_{i-3} + \Delta t \dfdx{\hat T_k^{-1}}{x_i}\omega_k^b & \text{if}3< \text{i} \leq 6 \\
   &col(\hat T_k^{-1},i-34) & \text{if } 31 < i \leq 34 \\
  \end{aligned} \right.
\]where
\begin{itemize}
\item $\dfdx{\hat T_k^-}{x_i}$ is the partial derivative of inverse of transformation matrix with respect to state [Appendix \ref{sec:avel_trfm}].
\end{itemize}

\item The prediction equation for the joint angles of multibody system is  $\hat{q}_{k+1}^- = \hat{q}_k + \Delta t \hat {\dot{q}}_k $. The Jacobian matrix of the equation is given by
\begin{equation}
\label{eq:dqdx}
\dfdx{\hat{q}_{k+1}^-}{x} = \left(\dfdx{\hat{q}_{k+1}^-}{x_{1}}, \dfdx{\hat{q}_{k+1}^-}{x_{2}}, \cdots , \dfdx{\hat{q}_{k+1}^-}{x_{62}}\right) \in \Re^{25 \times 62}
\end{equation}
\[
\dfdx{\hat{q}_{k+1}^-}{x_{i}} = 
	\begin{cases}
	l_{25,i} & \text{if } (6 < i \leq 31) \text{ or } (38 < i \leq 62) \\
	0 & \text{otherwise}   \\
	\end{cases}
\]where
\begin{itemize}
\item $l_{25,i}$ is a column vector of length 25 with 1 in the $i^{th}$ position and zeros in other position [Appendix \ref{sec:symbols}].
\end{itemize}

\item The prediction equation for the velocities of the robot is $\hat{\dot{y}}_{k+1}^- = \hat{\dot{y}}_{k}+ \Delta t \Lambda $, where 
$$\Lambda = M(\hat{y}_{k})^{-1}(-C(\hat{y}_{k},\hat{\dot{y}}_{k})\hat{\dot{y}}_{k} - g(\hat{y}_{k}) + J_r(\hat{y}_{k})^{T}W_{r,k+1} +J_l(\hat{y}_{k})^{T}W_{l,k+1} + \tau_{k+1})$$
The Jacobian matrix of the equation is given by
 \begin{equation}
 \label{eq:dydx}
\dfdx{\hat{\dot{y}}_{k+1}^-}{x} = \left(\dfdx{\hat{\dot{y}}_{k+1}^-}{x_{1}}, \dfdx{\hat{\dot{y}}_{k+1}^-}{x_{2}}, \cdots , \dfdx{\hat{\dot{y}}_{k+1}^-}{x_{62}}\right) \in \Re^{31 \times 62}
\end{equation}
where
\[
\dfdx{\hat{\dot{y}}_{k+1}^-}{x_{i}} = 
\left\{ 
\!\begin{aligned}
	& \left. \!\begin{aligned}
	-M_k^{-1}\dfdx{M_k}{x_{i}}\Lambda + &M_k^{-1}\left(-\dfdx{C_k}{x_{i}}\hat{\dot{y}}_{k} -\dfdx{g_k}{x_{i}}+ \right. \\
	&\left(\dfdx{J_{r,k}}{x_{i}}\right)^{T}W_{r,k+1} + \left. \left(\dfdx{J_{l,k}}{x_{i}}\right)^{T}W_{l,k+1} \right)
	\end{aligned} \right\}& \text{if } 0 < i \leq 31 \\
    & \left.\!\begin{aligned}
    l_{31,(i-31)}-&M_k^{-1}\dfdx{M_k}{x_{i}}\Lambda+ \\
    &M_k^{-1}\left(-\dfdx{C_k}{x_{i}}\hat{\dot{y}}_{k}- col(C_k,i-31)\right)     
    \end{aligned} \right\} & \text{if } i < 31  \\
\end{aligned}
\right.
\]where
\begin{itemize}
\item the shorthand form symbols used are  $M_k= M(\hat{y}_{k}),C_k=C(\hat{y}_{k},\hat{\dot{y}}_k),J_{r,k}=J_r(\hat{y}_{k}),J_{l,k}=J_l(\hat{y}_{k})$
\end{itemize}
\end{enumerate}
The system matrix $A_k$ is formulated by combining Equations \ref{eq:dpdx},\ref{eq:dthetadx},\ref{eq:dqdx} and \ref{eq:dydx} as
\begin{equation}
\label{eq:sys_mat}
A_k = \left(
\begin{aligned}
\dfdx{\hat{p}_{k+1}^-}{x} \\
\dfdx{\hat{\theta}_{k+1}^-}{x} \\
\dfdx{\hat{q}_{k+1}^-}{x}\\
\dfdx{\hat{\dot{y}}_{k+1}^-}{x}
\end{aligned} \right)
\in \Re^{62 \times 62}
\end{equation}

The noise correlation matrix $W_k$ is derived by discretizing the Equation \ref{eq:toro} and computing the Jacobian matrix of the discretized equation with respect to the noise $w$. Discretization of the Equation \ref{eq:toro} will result in a equation similar to \ref{eq:toro_dis}. The matrix $W_k$ is 
\begin{equation}
    \label{eq:toro_noisecorr}
    W_k =  \begin{bmatrix}
        \textbf{0}_{31,31} \\ 
        J_{r,k}^T \hspace{5mm} J_{l,k}^T \hspace{5mm} col(7:31,M_k^{-1})
        \end{bmatrix},
\end{equation}
where $\textbf{0}_{31,31}$ is a zero matrix of dimension $31 \times 31$, $col(7:31,M_k^{-1})$ represents the $7^{th}$ to ${31}^{st}$ colomns of the $M_k^{-1}$ matrix.

Substituting Equations \ref{eq:toro_dis}, \ref{eq:sys_mat} and \ref{eq:toro_noisecorr} in \ref{eq:predict} and substituting the values of process covariance $Q_k$ completes the prediction stage of EKF.

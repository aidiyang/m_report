\subsection{Update step}
\label{subsec:toro_update}
The update equation of the EKF is given in Equation \ref{eq:ekf_correct}. The measurement equation of the system is given by $$\hat{y}_{k+1} = h(\hat{x}_{k+1}^-,u_{k+1},0).$$ For the sake of simplicity let us assume the measurement of noise uncorrelated. That is the noise acting on a measurement does not depend on the noise acting on all other measurements. $$V_k = I_3.$$ Substituting the assumption in Equation \ref{eq:ekf_correct} gives
\begin{equation}
\label{eq:correct}
\begin{split}
K_{k+1} &= P_{k+1}^-\hat{C}_{k+1}^{T-}(\hat{C}_{k+1}^-P_{k+1}^-\hat{C}_{k+1}^{T-} + R_{k+1})^{-1}\\
\hat{x}_{k+1} &= \hat{x}_{k+1}^- + K_{k+1}(y_{k+1}-\hat{y}_{k+1})\\
P_{k+1} &= (I- K_{k+1}\hat{C}_{k+1}^-)P_{k+1}^-.
\end{split}
\end{equation}

For the computation of Kalman gain $K_k$ and to update the error covariance matrix $P_k$ in Equation \ref{eq:correct}, the measurement sensitivity matrix $C_k$ have to be computed. The matrix $C_k$ determined by computing the Jacobian matrix of the measurement equation \ref{eq:y_msr}. The computation of $C_k$ matrix is as follows:
\begin{enumerate}
\item The measurement equation for acceleration measurements is  $$\hat{acc}^b_{k+1} = \dot{v}^{b-}_{k+1}-\hat{R}_{k+1}^{T-}\begin{bmatrix} 0 \\ 0 \\ -9,81 \end{bmatrix}$$.
The Jacobian matrix is 
\begin{equation}
    \label{eq:dacc_msrdx}
    \dfdx{\hat{acc}_{k+1}^{b-}}{x} = \dfdx{\hat{\dot{v}}_{k+1}^{b-}}{x} - \dfdx{\hat{R}^{T-}_{k+1}}{x}\begin{bmatrix} 0 \\ 0 \\ -9,81 \end{bmatrix}  \in \Re^{3 \times 62}
\end{equation}where
\begin{itemize}
    \item $\dfdx{\hat{\dot{v}}_{k+1}^b}{x}$ is Jacobian matrix obtained by evaluating Equation \ref{eq:dydx} with the values $\hat x_{k+1}^-, u_{k+1}$ and picking the rows corresponding to the linear acceleration. For instance in Equation \ref{eq:dydx} the first three rows of the Jacobian matrix corresponds to the linear acceleration.
    \item $\dfdx{\hat{R}^T_{k+1}}{x}$ is partial derivative of Rotation matrix [Appendix \ref{sec:rot_mat}].
\end{itemize}

\item The Jacobian matrix corresponding to the angular velocity $\hat{\omega}^{b-}_{k+1}$ is 
\begin{equation}
    \label{eq:dw_msrdx} 
    \dfdx{\hat{\omega}^{b-}_{k+1}}{x} = \left(\dfdx{\hat{\omega}^{b-}_{k+1}}{x_{1}}, \dfdx{\hat{\omega}^{b-}_{k+1}}{x_{2}}, \cdots , \dfdx{\hat{\omega}^{b-}_{k+1}}{x_{62}}\right) \in \Re^{3 \times 62}
\end{equation}
\[ \dfdx{\hat{\omega}^{b-}_{k+1}}{x} = 
    \begin{cases}
    l_{3,i-34} & \text{if } 34 < i \leq 37 \\
    \textbf{0}_{3,1} &\text{otherwise}.
    \end{cases}
 \]where 
 \begin{itemize}  
 \item $l_{x,y}$ is a zero vector of length $x$ with 1 at the $y^{th}$ position.
 \item $\textbf{0}_{m,n}$ is the zero matrix of dimensions $m$ and $n$.
 \end{itemize}

\item The Jacobian matrix corresponding to the joint angles $\hat{q}_{k+1}^-$ is
\begin{equation}
\label{eq:dq_msrdx}
\dfdx{\hat{q}_{k+1}^-}{x} = \left(\dfdx{\hat{q}_{k+1}^-}{x_{1}}, \dfdx{\hat{q}_{k+1}^-}{x_{2}}, \cdots , \dfdx{\hat{q}_{k+1}^-}{x_{62}}\right) \in \Re^{25 \times 62}
\end{equation}
 \[
 \dfdx{\hat{q}_{k+1}^-}{x_{i}} =
 \begin{cases}
 l_{25,i-6} & \text{if } 6 < i \leq 31 \\
 \textbf{0}_{25,1} & \text{otherwise}.
 \end{cases}
 \]

\item The Jacobian matrix corresponding to the angular velocity of joint angles  $\hat{\dot{q}}_{k+1}^-$ is 
\begin{equation}
 \label{eq:ddq_msrdx}
\dfdx{\hat{\dot{q}}_{k+1}^-}{x} = \left(\dfdx{\hat{\dot{q}}_{k+1}^-}{x_{1}}, \dfdx{\hat{\dot{q}}_{k+1}^-}{x_{2}}, \cdots , \dfdx{\hat{\dot{q}}_{k+1}^-}{x_{62}}\right) \in \Re^{25 \times 62}
\end{equation}
  \[
 \dfdx{\hat{\dot{q}}_{k+1}^-}{x_{i}} =
 \begin{cases}
 l_{25,i-37} & \text{if } 37 < i \leq 62 \\
 \textbf{0}_{25,1} & \text{otherwise}.
 \end{cases}
 \]

 \item The measurement equation for the contact points in right and left foot are 
 $$ \hat{s}_{cnt,k+1}=
 \begin{bmatrix}
 \hat{s}_{r,k+1}^- \\ \hat{s}_{l,k+1}^-
 \end{bmatrix} 
 = \begin{bmatrix}
 \hat{H}_{r,k+1}^- s  \\ \hat{H}_{l,k+1}^- s
   \end{bmatrix}
	$$ 
where $s$ can be zero or more points in the set $ \lbrace s_A,s_B,s_C,s_D \rbrace$. The points in contact $s$ is determined by the Algorithm \ref{alg:cnt_switch}. 

 The Jacobian matrix of the contact measurements is 
\begin{equation}
    \label{eq:dscnt_msrdx}
    \begin{split}
    \dfdx{\hat{s}_{cnt,k+1}^-}{x} &= \left[
    \begin{aligned}
    \dfdx{\hat{s}_{r,k+1}^-}{x}  \\ \dfdx{\hat{s}_{l,k+1}^-}{x}
    \end{aligned} \right] = \left[
    \begin{aligned}
    \dfdx{\hat{H}_{r,k+1}^-}{x}s \\ \dfdx{\hat{H}_{l,k+1}^-}{x}s 
    \end{aligned}\right]
 \\ \vspace{5mm} \\
     \dfdx{\hat{H}_{f,k+1}^-}{x} = &\left( \dfdx{\hat{H}_{f,k+1}^-}{x_1}, \dfdx{\hat{H}_{f,k+1}^-}{x_2},\cdots, \dfdx{\hat{H}_{f,k+1}^-}{x_{62}} \right) \hspace{5mm} f \in \{r,l\}
     \end{split}
\end{equation}
where $\dfdx{\hat{H}_{f,k+1}^-}{x}$ is the partial derivative of homogeneous transformation matrix of a foot with respect to the system states [Appendix \ref{sec:htm}].
\end{enumerate}
The measurement sensitivity matrix $\hat{C}_{k+1}^-$ of the system is given by Equations \ref{eq:dacc_msrdx}, \ref{eq:dw_msrdx}, \ref{eq:dq_msrdx}, \ref{eq:ddq_msrdx}and \ref{eq:dscnt_msrdx}:
\begin{equation}
    \label{eq:msr_mat}
    \hat{C}^-_{k+1} = \left(
   \begin{aligned}
   \dfdx{\hat{acc}_{k+1}^{b-}}{x} \\
   \dfdx{\hat{\omega}^{b-}_{k+1}}{x} \\
    \dfdx{\hat{q}_{k+1}^-}{x} \\
    \dfdx{\hat{\dot{q}}_{k+1}^-}{x} \\
    \dfdx{\hat{s}_{cnt,k+1}^-}{x} \\
   \end{aligned}
	 \right) \in \Re^{80 \times 62}.
\end{equation}

\begin{comment}
\paragraph{Observability:}
State space representation of a linear system is,
\begin{equation}
\label{eq:dyn_l}
\begin{split}
\dot{x} &= Ax + Bu\\
y &= Cx + Du.
\end{split}
\end{equation}
where, $x \in \Re^{n}$ is the vector representing the states of the system. $u \in \Re^{p}$ is the vector of inputs, $y \in \Re^{m}$ is the vector of outputs of the system. $A \in \Re^{n \times n}$ is the system matrix. $B \in \Re^{n \times p}$ is the matrix relating state and input, $C \in \Re^{m \times n}$ is the measurement matrix relating output and state, $D \in \Re^{m \times p}$ is the matrix relating input and output of the system.

Linearising a nonlinear system in Equation \ref{eqn:nl_sys} at some operating point will lead to linear system of form Eq. \ref{eq:dyn_l}. For a linear system to be observable, it should satisfy
\begin{equation}
obs =
\begin{pmatrix}
C\\ CA \\ CA^{2}\\ \vdots \\ CA^{n-1}
\end{pmatrix}
, rank(obs) =n
\end{equation}
For our system to be observable $rank(obs) = 62$.
\end{comment}
%\textbf{ Make plots from files act=datsrc/ROBOT-TILT-0807.mat est=estimates-data/est-090701.mat or *080703.mat }
%\end{document}

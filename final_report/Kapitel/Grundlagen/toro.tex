\section{Toro}
\emph{Toro} has a complicated kinematic structure as shown in Figure \ref{fig:toro_kin}. It consists of 3 kinematic chains branching from the common base. The hip acts as the commnon base. The three chains branching from the hip are  upper body, right leg and left leg. The upper body further branches into right hand and left hand. The two arms and legs acts as the end effectors of \emph{Toro}.

\begin{figure}
\begin{center}
\includegraphics[trim= 70mm 10mm 40mm 10mm,clip,scale=0.7]{Bilder/TORO_kinematic.pdf}
\caption{Kinematic chain of \emph{Toro} \underline{Explain the joints symbols}}
\label{fig:toro_kin}
\end{center}
\end{figure}

The generalized coordinates of \emph{Toro} are represneted by a vector of joint angles $q$ and the coordinates of the base of robot $x^b$. Number of joints of a robot is the number of controllable degrees of freedom \cite[Chapter 2]{mur94}. The coordinates of the base $x^b$ are the underactuated degrees of freedom of \emph{Toro}. 

A rigid body in space has six degrees of freedom as shown in Figure \ref{fig:rbody}.
%\begin{figure}
\begin{figure}
\begin{center}
\includegraphics[trim= 30mm 100mm 10mm 120mm,scale=0.75]{Bilder/rbody_dof.pdf}
\caption[Degrees of freedom of a rigid body]{Degrees of freedom of a rigid body \footnotemark[1]}
\label{fig:rbody}
\end{center}
\end{figure}
The translational and rotational degrees of freedom of the rigid body in Figure \ref{fig:rbody} are \emph{x,y,z} and \emph{roll,pitch,yaw}. The rotational degrees of freedom \emph{roll,pitch,yaw} corresponds to rotation of the body around \textbf{X,Y,Z} axes. The base of \emph{Toro} is assumed to be a free rigid body. Since the base has all the six degrees of freedom which corresponds to degrees of freedom of rigid body floating in three dimensional space, it is called floating base. The equation of motion of a single rigid body(floating base) is given by Newton-Euler equation of motion in body coordinates \cite[Chapter 4]{mur94}. 
\footnotetext[1]{Image source:\url{http://www.cncexpo.com/Images/pitchyawroll.jpg}}
\begin{equation}
\label{eq:dyn_rig_bdy}
\begin{bmatrix}
mI_3 & \textbf{0}_3 \\ \textbf{0}_3 &\Im
\end{bmatrix}
\begin{bmatrix}
\dot{v}^b\\ \dot{\omega}^b
\end{bmatrix}
+ \begin{bmatrix}
\omega^b \times mv^b \\ 
\omega^b \times \Im w^b
\end{bmatrix}
= W^b.\\
\end{equation}
In the above equation, \emph{m} is the mass of the rigid body, $\Im$ is the moment of inertia of the rigid body. $I_3$ represents the $3 \times 3$ identity matrix and $ \textbf{0}_3$ represents the  $3 \times 3$ zero matrix. $[\dot{v}^b,\dot{\omega}^b]$ are the translational velocity and angular velocity represented in body coordinate frame. $W^b$ is the body wrench applied to the center of mass of body. The body wrench represents a force/moment pair acting on the body. It is represented by vector in $\Re^6$ as \citep{mur94}
$$ W^b = \begin{bmatrix} F \\ \tau \end{bmatrix}, $$ where $F \in \Re^3$ is the linear component and $\tau \in \Re^3$ is the rotational component of the genaralized force.

To be consistent with the multibody system dynamics formulation in Equation \ref{eq:dyn_mul_bdy} Newton Euler Equation \ref{eq:dyn_rig_bdy} is reformulated as
\begin{equation}
\label{eq:dyn_rig_bdy_sh}
M_x^b \dot V^b + C_x^b V^b+g_x^b = W^b,
\end{equation}
 where $x^b=\begin{bmatrix}p \\ \theta\end{bmatrix} \in \Re^6$ is a vector representing the position and orientation of the rigid body with respect to world coordinate frame or spatial frame. The three dimensional position vector is $p=\begin{bmatrix}p_x \\ p_y \\ p_z \end{bmatrix} \in \Re^3$ and the three dimesional orientation vector is $ \theta= \begin{bmatrix} \theta_{x} \\ \theta_{y} \\ \theta_{z} \end{bmatrix} \in \Re^3$. The orientation of the rigid body is parmeterised by Euler angles [Appendix \ref{eq:rot_full}]. 
 The body velocity $V^b$ is 
 \begin{equation}
\label{eq:body_vel}
V^b =
\begin{bmatrix}
v^b \\ \omega^b
\end{bmatrix}
= \begin{bmatrix}
R^T \dot{p} \\ (R^T \dot{R})^\vee
\end{bmatrix},
\end{equation}
where $\vee$ operator denotes the extraction of 3 dimensional vector from a skew symmetric matrix[Appendix \ref{sec:avel_trfm}]. The rotation matrix $R$ is the matrix dependent on parameterization of Euler angles [Appendix \ref{eq:rot_full}]. The acceleration $\dot{V}^b$ is the acceleration with respect to body frame. $M_x^b$ represents the inertia matrix of the rigid body in body coordinates. $C_x^b$ is the matrix representing the Coriolis and centrifugal forces acting on the system in body coordinates. $g_x^b$ is the vector representing the gravitational forces acting on the body in body coordinates. $W^b$ is the external force applied to the center of mass of the body. \footnote[2]{Inertia, Coriolis and gravity are assumed to be given with respect to the body coordinate frame for the rest of the report}.

The equations of motion of \emph{Toro} in Figure \ref{fig:toro_kin} is composed of equation of motion of the multibody system and equation of motion of single rigid body. The equation of motion of \emph{Toro} is formulated like the equation of motion for a floating base system as given in \cite{ott09}.
\begin{equation} \label{eq:dyn_biped}
\begin{bmatrix}
M_x &M_{xq} \\ M_{qx} &M_q
\end{bmatrix}
\begin{bmatrix}
\ddot{x}^b \\ \ddot{q}
\end{bmatrix}
+
\begin{bmatrix}
C_x &C_{xq} \\ C_{qx} &C_q
\end{bmatrix}
\begin{bmatrix}
\dot{x}^b \\ \dot{q}
\end{bmatrix}
+
\begin{bmatrix}
g_x \\ g_q
\end{bmatrix}
=
\begin{bmatrix}
0 \\ \tau
\end{bmatrix}
+ (J_r^b)^T W_r^b + (J_l^b)^T W_l^b,
\end{equation}
where the terms with subscript $x$ are the parameters of floating  base, the terms with subscript $q$ are the parameters of the multibody system and the terms with subscript $xq, qx$ are the coupling terms that connects the dynamics of the floating base with dynamics of the multibody system.

Equation \ref{eq:dyn_biped} can be written in a simplified form as
\begin{equation} \label{eq:dyn_sbiped}
M(y)\ddot{y} + C(y,\dot{y})\dot{y} + g(y) = \tau + (J_r^b)^T W_r^b + (J_l^b)^T W_l^b,
\end{equation}
where $y = \begin{bmatrix} x^b \\ q \end{bmatrix}$ is the vector representing the state variables of \emph{Toro}. The generalized coordinates of multibody system is $q \in \Re^{25}$ which represents the vector of joint angles. The vector of generalized velocities is $\dot{y}=\begin{bmatrix} V^{b} \\ \dot{q} \end{bmatrix} \in \Re^{31}.$ The vector of generalized accelerations is $\ddot{y}\in \Re^{31}.$  $M(y)\in \Re^{31 \times 31}$ is the inertia matrix, $C(y,\dot{y})\in \Re^{31 \times 31}$ is the matrix accounting for centrifugal and Coriolis forces. $g(y) \in \Re^{31}$ is the gravity vector. $\tau \in \Re^{31}$ is the vector of actuating torques acting on the robot, where the first six components are zero because those degrees of freedom corresponding to $x_f$ are not actuated. $J_r^b,J_l^b \in \Re^{31 \times 6}$ are the body Jacobian matrices that transforms the wrenches $W_r^b,W_l^b \in \Re^{6}$ applied in the right and left foot to generalized forces acting on the robot. From here on the super script $b$ in body Jacobian matrix and body wrenches is omitted for the sake of simplicity.

The forward dynamics equation of \emph{Toro} is
\begin{equation}
	\label{eq:motion}
	\ddot{y} = M(y)^{-1}(-C(y,\dot{y})\dot{y} - g(y) + J_r(y)^{T}W_{r} +J_l(y)^{T}W_{l} + \tau). 
\end{equation}

\subsection{State space representation:}
The state space representation of \emph{Toro} is formulated similar to the state space representation of multibody system in Equation \ref{eq:dyn_ss}. The state space representation of Equation \ref{eq:motion} is
\begin{equation}
\label{eq:newton_motion}
 \begin{bmatrix}
\frac{dy}{dt} \\ \frac{d\dot y}{dt}
\end{bmatrix}
= \begin{bmatrix}
\dot y \\  M(y)^{-1}(-C(y,\dot{y})\dot{y} - g(y) + J_r(y)^{T}W_{r} +J_l(y)^{T}W_{l} + \tau) 
\end{bmatrix}
\end{equation}

where $\dot y = \begin{bmatrix} V^b \\ \dot q \end{bmatrix}$. The body velocity $V^b$ is given in Equation \ref{eq:body_vel}. In order for th system to be integrable the linear part of the body velocity is rearranged as 
\begin{equation}
    \label{eq:transfo_linvel}
    \dot p = R v^b.
\end{equation}
Likewise the angular part of the body velocity can be reformulated as 
\begin{equation}
    \label{eq:transfo_angvel}
    \omega^b = T(\theta)\dot{\theta},
\end{equation}
where $T(\theta)$ is the transformation matrix [Appendix \ref{sec:avel_trfm}]. 

 The state space representation of \emph{Toro} is obtained by substituting Equations \ref{eq:transfo_linvel} and \ref{eq:transfo_angvel} in Equation \ref{eq:newton_motion}.
\begin{equation}
\label{eq:toro}
	\dot{x} = 
	\begin{bmatrix}
	\dot{p} \\ \dot{\theta} \\ \dot{q} \\ \ddot{y}
	\end{bmatrix}
	=
	\begin{bmatrix}
	R v^b\\	
	T(\theta)^{-1} \omega_f^b \\
	\dot{q}\\
	M(y)^{-1}(-C(y,\dot{y})\dot{y} -g(y) +  J_r(y)^{T}W_{r} +J_l(y)^{T}W_{l} + \tau)
	\end{bmatrix}.
	\\
\end{equation}	
\begin{comment}
\begin{itemize}
\item $$ y = \begin{bmatrix} p \\ \theta \\ q \end{bmatrix} = \begin{bmatrix} x_f \\ q \end{bmatrix}, \dot{y} = \begin{bmatrix} V^b \\ \dot{q}\end{bmatrix}, x = \begin{bmatrix}y \\ \dot{y}\end{bmatrix} $$  $x_f,q$ are the parameters of the floating base and joints as described in \ref{eq:motion}. $V^b$ is the body velocity as defined in \ref{eq:body_vel} and $\dot{q}$ is the velocities of the joints of the robot. \emph{x} is the vector of system states.
\item $T(\theta_{f})$ is the matrix that transforms the angular velocity $\omega_{f}^{b}$ to the time derivative of Euler angles $\dot{\theta}_{f}$. i.e $\omega_{f}^{b}=T(\theta_{f}) \dot{\theta}_{f}$. 
\item R is the rotation matrix which describes the rotation of floating base with respect to spatial frame. $ R = R_x(\theta_x) R_y(\theta_y) R_z(\theta_z)$
\end{itemize}
\end{comment}

%\section{Prediction step}
\subsection{Prediction step}
\label{subsec:toro_predict}
The prediction equations of EKF given in Equation \ref{eq:ekf_predict} are as follows:
\begin{equation}
\label{eq:predict}
\begin{split}
\hat{x}_{k+1}^- &= f(\hat{x}_{k},u_{k+1},0)\\
P_{k+1}^- &= A_kP_{k}A_k^T + W_kQ_{k}W_k^T \\
\end{split}
\end{equation}
The state space model of \emph{Toro} in \ref{eq:toro} is discretized for the implementation of EKF. For smaller integration time steps $\Delta t = 1ms$ the forward Euler discretization method can be used to discretize the continuous time model. The forward Euler discretization equation is $$ x_{k+1} = x_k + \Delta t f(x),$$ where $f(x)=\dot x$ is the nonlinear function describing the system. The discretized prediction equation of \emph{Toro} is obtained by substituting $0$ for noise terms $w_\tau, w_{W_r}$ and $w_{W_l}$ in Equation \ref{eq:toro} and applying the forward Euler discretization:
\begin{equation}
\label{eq:toro_dis}
	\begin{bmatrix}
	\hat{p}_{k+1}^- \\ \hat{\theta}_{k+1}^- \\ \hat{q}_{k+1}^- \\ \hat{\dot{y}}_{k+1}^-
	\end{bmatrix}
	 =   
	 \begin{bmatrix}
	 \hat{p}_k \\ \hat{\theta}_k \\ \hat{q}_k \\ \hat{\dot{y}}_{k}
	\end{bmatrix}	  
	+ \Delta t f(\hat{x}_k,u_{k+1}) \\
\end{equation}
$$ f(\hat{x}_k,u_{k+1})\footnotemark[1] = 
	\begin{bmatrix}
	\hat R_k \hat v^b_k \\
	\hat T_k^{-1} \hat \omega_k^b  \\
	\hat{\dot{q}}_k\\
	M(\hat{y}_{k})^{-1}(-C(\hat{y}_{k},\hat{\dot{y}}_{k})\hat{\dot{y}}_{k} -g(\hat{y}_{k}) +  J_r(\hat{y}_{k})^{T}W_{r,k+1} +J_l(\hat{y}_{k})^{T}W_{l,k+1} + \tau_{k+1})	
	\end{bmatrix}  $$
where $\hat{x}(t_k) = \hat{x}(k \Delta t) = \hat{x}_k$ represents the state x at \emph{kth} sampling instant. $\hat{x}_{k+1} = \hat{x}(k \Delta t + \Delta t)$ represents the state of the system at the next sampling instant. $u_{k+1}$ is the input at sampling instant $k+1$. The inputs $\tau, W_l, W_r$ remains constant in the interval between two sampling instant because of the zero order hold mechanism in the sensor. The above equation is used to predict the state $\hat{x}_k$ ahead of time in Equation \ref{eq:predict}. 
\footnotetext[1]{$\hat R_k, \hat T_k$ are the shorthand notations for $R(\hat \theta_k)$ and $T(\hat \theta_k)$ in the equation.}

For the computation of state covariance matrix $P_k^-$ in Equation \ref{eq:predict}, the Jacobian matrix $A$ should be computed. The Jacobian matrix is the matrix of first order partial derivatives of a vector valued function \citep{wal76}.

The Jacobian matrix computation of the prediction Equations \ref{eq:toro_dis} of EKF is as follows:
\begin{enumerate}
\item The prediction equation for the position of the base is $ \hat{p}_{k+1}^- = \hat{p}_k + \Delta t \hat R_k \hat v^b_k$, where  $ \hat{p}_k = [\hat{p}_{x,k},\hat{p}_{y,k},\hat{p}_{z,k}]$ is the coordinates of position at time instant $k$. The Jacobian matrix of the equation is given by
\begin{equation}
\label{eq:dpdx}
\dfdx{\hat{p}_{j,k+1}^-}{x} = \left(\dfdx{\hat{p}_{j,k+1}^-}{x_{1}}, \dfdx{\hat{p}_{j,k+1}^-}{x_{2}}, \cdots , \dfdx{\hat{p}_{j,k+1}^-}{x_{62}}\right) \in \Re^{3 \times 62} \hspace{2cm} j=1,2,3
\end{equation}
\[
 \dfdx{\hat{p}_{k+1}^-}{x_{i}} =  \left\lbrace
  \!\begin{aligned}
   &e_i & \text{if }(i=j)\\
   &\Delta t \dfdx{\hat R_k}{x_i} \hat v^b_k & \text{if }(3 < i \leq 6)\\
   &\textbf{0}_{3 \times 1} &\text{if }(6 < i \leq 31) \text{ or } (35 < i \leq 62) \\
   &col(\hat R_k,i-31) & \text{if } 31 < i \leq 34 \\
  \end{aligned} \right.
\]where
\begin{itemize}
\item the subscript \emph{j} represents the row dimension and \emph{i} represents the column dimension of the Jacobian matrix,
\item $col(X,i)$ - represents the $i^{th}$ column of matrix $X$,
\item the partial derivative of \emph{R} with respect to the state $\hat{x}_k$ is $\dfdx{\hat R_k}{x_i}$ [Appendix \ref{sec:rot_mat}]),
\item  $e_i$ is the unit vectors in direction of coordinate axis and  $\textbf{0}_{3 \times 1}$ is the zero vector of dimensions 3 [Appendix \ref{sec:symbols}].
\end{itemize}

\item The prediction equation for the orientation of the base is $\hat{\theta}_{k+1}^- = \hat{\theta}_k + \Delta t \hat T_k^{-1} \omega_k^b$, where $\hat \theta_k = [ \theta_{x,k},\theta_{y,k},\theta_{z,k}]$ are the coordinates of orientation of the base. The Jacobian matrix of the equation is given by
\begin{equation}
\label{eq:dthetadx}
\dfdx{\hat{\theta}_{j,k+1}^-}{x} = \left(\dfdx{\hat{\theta}_{j,k+1}^-}{x_{1}}, \dfdx{\hat{\theta}_{j,k+1}^-}{x_{2}}, \cdots , \dfdx{\hat{\theta}_{j,k+1}^-}{x_{62}}\right) \in \Re^{3 \times 62}
\end{equation}
\[
 \dfdx{\hat{\theta}_{k+1}^-}{x_{i}} = \left\lbrace
  \!\begin{aligned}
   &\textbf{0}_{3 \times 1} &\text{if }(0 < i \leq 3) \text{ or }(6 < i \leq 31) \\
   &\hspace{2cm} &\text{ or } (35 < i \leq 62) \\
   &e_{i-3} + \Delta t \dfdx{\hat T_k^{-1}}{x_i}\omega_k^b & \text{if}3< \text{i} \leq 6 \\
   &col(\hat T_k^{-1},i-34) & \text{if } 31 < i \leq 34 \\
  \end{aligned} \right.
\]where
\begin{itemize}
\item $\dfdx{\hat T_k^-}{x_i}$ is the partial derivative of inverse of transformation matrix with respect to state [Appendix \ref{sec:avel_trfm}].
\end{itemize}

\item The prediction equation for the joint angles of multibody system is  $\hat{q}_{k+1}^- = \hat{q}_k + \Delta t \hat {\dot{q}}_k $. The Jacobian matrix of the equation is given by
\begin{equation}
\label{eq:dqdx}
\dfdx{\hat{q}_{k+1}^-}{x} = \left(\dfdx{\hat{q}_{k+1}^-}{x_{1}}, \dfdx{\hat{q}_{k+1}^-}{x_{2}}, \cdots , \dfdx{\hat{q}_{k+1}^-}{x_{62}}\right) \in \Re^{25 \times 62}
\end{equation}
\[
\dfdx{\hat{q}_{k+1}^-}{x_{i}} = 
	\begin{cases}
	l_{25,i} & \text{if } (6 < i \leq 31) \text{ or } (38 < i \leq 62) \\
	0 & \text{otherwise}   \\
	\end{cases}
\]where
\begin{itemize}
\item $l_{25,i}$ is a column vector of length 25 with 1 in the $i^{th}$ position and zeros in other position [Appendix \ref{sec:symbols}].
\end{itemize}

\item The prediction equation for the velocities of the robot is $\hat{\dot{y}}_{k+1}^- = \hat{\dot{y}}_{k}+ \Delta t \Lambda $, where 
$$\Lambda = M(\hat{y}_{k})^{-1}(-C(\hat{y}_{k},\hat{\dot{y}}_{k})\hat{\dot{y}}_{k} - g(\hat{y}_{k}) + J_r(\hat{y}_{k})^{T}W_{r,k+1} +J_l(\hat{y}_{k})^{T}W_{l,k+1} + \tau_{k+1})$$
The Jacobian matrix of the equation is given by
 \begin{equation}
 \label{eq:dydx}
\dfdx{\hat{\dot{y}}_{k+1}^-}{x} = \left(\dfdx{\hat{\dot{y}}_{k+1}^-}{x_{1}}, \dfdx{\hat{\dot{y}}_{k+1}^-}{x_{2}}, \cdots , \dfdx{\hat{\dot{y}}_{k+1}^-}{x_{62}}\right) \in \Re^{31 \times 62}
\end{equation}
where
\[
\dfdx{\hat{\dot{y}}_{k+1}^-}{x_{i}} = 
\left\{ 
\!\begin{aligned}
	& \left. \!\begin{aligned}
	-M_k^{-1}\dfdx{M_k}{x_{i}}\Lambda + &M_k^{-1}\left(-\dfdx{C_k}{x_{i}}\hat{\dot{y}}_{k} -\dfdx{g_k}{x_{i}}+ \right. \\
	&\left(\dfdx{J_{r,k}}{x_{i}}\right)^{T}W_{r,k+1} + \left. \left(\dfdx{J_{l,k}}{x_{i}}\right)^{T}W_{l,k+1} \right)
	\end{aligned} \right\}& \text{if } 0 < i \leq 31 \\
    & \left.\!\begin{aligned}
    l_{31,(i-31)}-&M_k^{-1}\dfdx{M_k}{x_{i}}\Lambda+ \\
    &M_k^{-1}\left(-\dfdx{C_k}{x_{i}}\hat{\dot{y}}_{k}- col(C_k,i-31)\right)     
    \end{aligned} \right\} & \text{if } i < 31  \\
\end{aligned}
\right.
\]where
\begin{itemize}
\item the shorthand form symbols used are  $M_k= M(\hat{y}_{k}),C_k=C(\hat{y}_{k},\hat{\dot{y}}_k),J_{r,k}=J_r(\hat{y}_{k}),J_{l,k}=J_l(\hat{y}_{k})$
\end{itemize}
\end{enumerate}
The system matrix $A_k$ is formulated by combining Equations \ref{eq:dpdx},\ref{eq:dthetadx},\ref{eq:dqdx} and \ref{eq:dydx} as
\begin{equation}
\label{eq:sys_mat}
A_k = \left(
\begin{aligned}
\dfdx{\hat{p}_{k+1}^-}{x} \\
\dfdx{\hat{\theta}_{k+1}^-}{x} \\
\dfdx{\hat{q}_{k+1}^-}{x}\\
\dfdx{\hat{\dot{y}}_{k+1}^-}{x}
\end{aligned} \right)
\in \Re^{62 \times 62}
\end{equation}

The noise correlation matrix $W_k$ is derived by discretizing the Equation \ref{eq:toro} and computing the Jacobian matrix of the discretized equation with respect to the noise $w$. Discretization of the Equation \ref{eq:toro} will result in a equation similar to \ref{eq:toro_disc}. The matrix $W_k$ is 
\begin{equation}
    \label{eq:toro_noisecorr}
    W_k =  \begin{bmatrix}
        \textbf{0}_{31,31} \\ 
        J_{r,k}^T \hspace{5mm} J_{l,k}^T \hspace{5mm} col(7:31,M_k^{-1})
        \end{bmatrix},
\end{equation}
where $\textbf{0}_{31,31}$ is a zero matrix of dimension $31 \times 31$, $col(7:31,M_k^{-1})$ represents the $7^{th}$ to ${31}^{st}$ colomns of the $M_k^{-1}$ matrix.

Substituting Equations \ref{eq:toro_dis}, \ref{eq:sys_mat} and \ref{eq:toro_noisecorr} in \ref{eq:predict} and substituting the values of process covariance $Q_k$ completes the prediction stage of EKF.


%\section{Update step}
\subsection{Update step}
\label{subsec:toro_update}
The update equation of the EKF is given in Equation \ref{eq:ekf_correct}. The measurement equation of the system is given by $$\hat{y}_{k+1} = h(\hat{x}_{k+1}^-,u_{k+1},0).$$ For the sake of simplicity let us assume the measurement of noise uncorrelated. That is the noise acting on a measurement does not depend on the noise acting on all other measurements. $$V_k = I_3.$$ Substituting the assumption in Equation \ref{eq:ekf_correct} gives
\begin{equation}
\label{eq:correct}
\begin{split}
K_{k+1} &= P_{k+1}^-\hat{C}_{k+1}^{T-}(\hat{C}_{k+1}^-P_{k+1}^-\hat{C}_{k+1}^{T-} + R_{k+1})^{-1}\\
\hat{x}_{k+1} &= \hat{x}_{k+1}^- + K_{k+1}(y_{k+1}-\hat{y}_{k+1})\\
P_{k+1} &= (I- K_{k+1}\hat{C}_{k+1}^-)P_{k+1}^-.
\end{split}
\end{equation}
The measurements of \emph{Toro} are Cartesian accelerations($acc^b$) of the hip measured by accelerometer, angular velocity($\omega^b$) of the hip measured by the gyroscope, joint angles($q_j$) and joint velocities($\dot{q_j}$) measured by joint encoders.
\begin{equation}
    \label{eq:y_sens}
     y_{sens} = \begin{bmatrix} acc^b \\ \omega^b \\ q_j \\ \dot{q}_j \end{bmatrix} 
\end{equation}
\begin{itemize}
    \item The simplified model of $IMU_{acc}$ is $$ acc^b= \begin{bmatrix} acc^b_x \\ acc^b_y \\ acc^b_z \end{bmatrix} =  \begin{bmatrix}\ddot{p}_x^b \\ \ddot{p}_y^b \\ \ddot{p}_z^b \end{bmatrix} - R^T \begin{bmatrix}0 \\0 \\-9.81 \end{bmatrix}$$ \emph{R} is the rotational matrix that transforms a vector in body coordinate frame to spatial frame.
    \item $\ddot{p}^b$ is computed from the forward dynamic equation \ref{eq:motion} using the predicted values of the state
    \item $\omega_{f}^{b} $ is the vector of angular rates of the hip(floating base) measured by gyroscope. The measurements are in the frame attached to the hip. 
\end{itemize}
\begin{figure}
    \begin{center}
    %trim option's parameter order: left bottom right top
    \includegraphics[trim= 20mm 150mm 20mm 50mm,scale=0.80]{Bilder/foot_topview.pdf}
    \caption{Toro feet viewed from top}
    \label{fig:biped_feet}
    \end{center}
\end{figure}
Along with the sensor measurements kinematic constraints can also be introduced as measurements \citep{atk12}. The kinematic constraint that is considered as measurement is the position constraint. For instance when the robot is tilting around an edge, the position of the points that are lying on that edge does not change with respect to the spatial frame (world frame). 
 
 Figure \ref{fig:biped_feet} shows the contact points considered for measurements. The corner points of each foot are measured with respect to spatial frame \emph{S}. The measuremnt equation of these contact points are
\begin{equation}
    \label{eq:y_kin}
    \begin{split}
    y_{kin} &= s_{contact} = \begin{bmatrix}s_{r}\\ s_{l}\end{bmatrix},\\
    \end{split}
\end{equation} where
$s_{r},s_{l}$ are the vectors of contact points in right foot and left foot defined with respect to spatial frame. They are given as
\begin{equation}
    \begin{split}
    s_{r} &= \begin{bmatrix} s_{A,r}\\ s_{B,r}\\ s_{C,r}\\ s_{D,r}\end{bmatrix}= \begin{bmatrix} {H}_{r}s_{A}\\  {H}_{r}s_{B}\\  {H}_{r}s_{C}\\  {H}_{r}s_{D}\end{bmatrix} \\
    s_{l} &= \begin{bmatrix} s_{A,l}\\ s_{B,l}\\ s_{C,l}\\ s_{D,l}\end{bmatrix}= \begin{bmatrix} {H}_{l}s_{A}\\  {H}_{l}s_{B}\\  {H}_{l}s_{C}\\  {H}_{l}s_{D}\end{bmatrix} \\
    \end{split}
\end{equation}
where ${H}_{r},{H}_{l}$ are the homogeneous transformation matrices of the right and left foot with respect to spatial frame [Appendix \ref{sec:htm}]
 In Figure \ref{fig:biped_feet} $s_A,s_B,s_C,s_D$ are constant with resptect to the foot coordinate frames \emph{RF,LF}. They are given as 
 $$ s_A = \begin{bmatrix} 0.13 \\ 0.0475 \\ 0 \end{bmatrix} , 
    s_B = \begin{bmatrix} 0.13 \\ -0.0475 \\ 0 \end{bmatrix},
    s_C = \begin{bmatrix} -0.06 \\ -0.0475 \\ 0 \end{bmatrix} \text{ and }
    s_D = \begin{bmatrix} -0.06 \\ 0.0475 \\ 0 \end{bmatrix}.$$

\paragraph{Switching:} The measurement of the contact points with respect to the spatial frame \emph{S} are made before starting the experiment. The measurements are made under the circumstance where both feet of the robot is in contact with the ground. These measurements are assumed to remain constant throughout the experiment. When the robot is standing on the ground, all the contact measurements are valid. But when the robot starts to tilt around an edge, some of the point measurements becomes invalid. For instance let us consider standing on right leg, when the robot is tilting around the back edge, the points $s_{C}$ and $s_{D}$ remain in contact with the ground, whereas the points $s_{A}$ and $s_B$ lifts off from the floor. The measurements from these points are not valid. It is important to detect the situations when the robot is standing still and when it is tilting around an edge, so that we consider only the valid measurements. Zero moment point (ZMP) is an useful criteria for detecting the tilting edge. The ZMP of a foot is computed from the ground reaction force measured by the FTS [Appendix \ref{sec:zmp}]. The contact measurements also differ when the a single foot is in contact (single support) or two feet are in contact(double support). The single and double support cases can be differentiate by measuring the vertical ground reactional force $F_z$. The contact case is determined by the parameter $\alpha$ which is given as \citep{atk12} $$ \alpha = \frac{F_{z,r} + F_{z,l}}{F_{z,r}}.$$ 

\begin{algorithm}
    \caption{Selection of contact measurements}
    \label{alg:cnt_switch}
    \begin{algorithmic}
    %\REQUIRE  $s_{cnt}$
    %\ENSURE $FOO$
    \IF {$\alpha = 0.5$}
    %{Double support}
        \IF {$ {zmp}_x \geq 0.12$}
        % Tilting around front edge
        \RETURN $ s_{cnt}= \begin{bmatrix}s_{A,r} &s_{B,r} &s_{A,l} &s_{B,l} \end{bmatrix}^T$ 
        \ELSIF {${zmp}_x \leq -0.05$}
        % Tilting around back edge
        \RETURN $ s_{cnt}= \begin{bmatrix}s_{C,r} &s_{D,r} &s_{C,l} &s_{D,l} \end{bmatrix}^T$ 
        \ENDIF
    \ELSIF {$\alpha = 1$}
    % Support right foot
        \IF {$ {zmp}_x \geq 0.12$}
        % Tilting around front edge
        \RETURN $ s_{cnt}= \begin{bmatrix}s_{A,r} &s_{B,r}  \end{bmatrix}^T$ 
        \ELSIF {${zmp}_x \leq -0.05$}
        % Tilting around back edge
        \RETURN $ s_{cnt}= \begin{bmatrix}s_{C,r} &s_{D,r}  \end{bmatrix}^T$ 
        \ELSIF {$ {zmp}_y \geq 0.04$}
        % Tilting around inner edege
        \RETURN $ s_{cnt}= \begin{bmatrix}s_{A,r} &s_{D,r}  \end{bmatrix}^T$ 
        \ELSIF {${zmp}_y \leq -0.04$}
        % Tilting around outer edge
        \RETURN $ s_{cnt}= \begin{bmatrix}s_{B,r} &s_{C,r}  \end{bmatrix}^T$ 
        \ENDIF
    \ELSIF {$\alpha = 0$}
    % Support left foot
        \IF {$ {zmp}_x \geq 0.12$}
        % Tilting around front edge
        \RETURN $ s_{cnt}= \begin{bmatrix} s_{A,l} &s_{B,l} \end{bmatrix}^T$ 
        \ELSIF {${zmp}_x \leq -0.05$}
        % Tilting around back edge
        \RETURN $ s_{cnt}= \begin{bmatrix} s_{C,l} &s_{D,l} \end{bmatrix}^T$ 
        \ELSIF {$ {zmp}_y \geq 0.04$}
        % Tilting around outer edege
        \RETURN $ s_{cnt}= \begin{bmatrix}s_{A,r} &s_{D,r}  \end{bmatrix}^T$ 
        \ELSIF {${zmp}_y \leq -0.04$}
        % Tilting around inner edge
        \RETURN $ s_{cnt}= \begin{bmatrix}s_{B,r} &s_{C,r}  \end{bmatrix}^T$ 
        \ENDIF
    \ENDIF
    \end{algorithmic}
\end{algorithm}

The Algorithm \ref{alg:cnt_switch} is useful in making the decision about the contact points $s_{cnt}$ based on number of support and ZMP. The ZMP is defined with respect to the foot coordinate frame. For instance the front edge of right foot in Figure \ref{fig:biped_feet} is $0.13m$ from the foot coordinate frame $RL$. When the ${zmp}_x$ of the foot reaches this value then the foot starts to tilt around the front edge. Like wise the foot starts tilting backwards when the ${zmp}_x$ reaches the value $-0.06$. Likewise the upper and lower bounds of $zmp_y$ are $0.0475$ and $-0.0475$.

The full measurement equations of the system is obtained by combining Equations \ref{eq:y_sens} and \ref{eq:y_kin}. The discretized form of measurement equations is
\begin{equation}
    \label{eq:y_msr}
    y_{k+1} = \begin{bmatrix} y_{sens,k+1} \\ y_{kin,k+1} \end{bmatrix}= \begin{bmatrix} acc^b_{k+1} \\ \omega^b_{k+1} \\ q_{k+1} \\ \dot q_{k+1} \\ s_{cnt,k+1} \end{bmatrix}.
\end{equation}

For the computation of Kalman gain $K_k$ and to update the error covariance matrix $P_k$ in Equation \ref{eq:correct}, the measurement sensitivity matrix $C_k$ have to be computed. The matrix $C_k$ determined by computing the Jacobian matrix of the measurement equation \ref{eq:y_msr}. The computation of $C_k$ matrix is as follows:
\begin{enumerate}
\item The measurement equation for acceleration measurements is  $$\hat{acc}^b_{k+1} = \dot{v}^{b-}_{k+1}-\hat{R}_{k+1}^{T-}\begin{bmatrix} 0 \\ 0 \\ -9,81 \end{bmatrix}$$.
The Jacobian matrix is 
\begin{equation}
    \label{eq:dacc_msrdx}
    \dfdx{\hat{acc}_{k+1}^{b-}}{x} = \dfdx{\hat{\dot{v}}_{k+1}^{b-}}{x} - \dfdx{\hat{R}^{T-}_{k+1}}{x}\begin{bmatrix} 0 \\ 0 \\ -9,81 \end{bmatrix}  \in \Re^{3 \times 62}
\end{equation}where
\begin{itemize}
    \item $\dfdx{\hat{\dot{v}}_{k+1}^b}{x}$ is Jacobian matrix obtained by evaluating Equation \ref{eq:dydx} with the values $\hat x_{k+1}^-, u_{k+1}$ and picking the rows corresponding to the linear acceleration. For instance in Equation \ref{eq:dydx} the first three rows of the Jacobian matrix corresponds to the linear acceleration.
    \item $\dfdx{\hat{R}^T_{k+1}}{x}$ is partial derivative of Rotation matrix [Appendix \ref{sec:rot_mat}].
\end{itemize}

\item The Jacobian matrix corresponding to the angular velocity $\hat{\omega}^{b-}_{k+1}$ is 
\begin{equation}
    \label{eq:dw_msrdx} 
    \dfdx{\hat{\omega}^{b-}_{k+1}}{x} = \left(\dfdx{\hat{\omega}^{b-}_{k+1}}{x_{1}}, \dfdx{\hat{\omega}^{b-}_{k+1}}{x_{2}}, \cdots , \dfdx{\hat{\omega}^{b-}_{k+1}}{x_{62}}\right) \in \Re^{3 \times 62}
\end{equation}
\[ \dfdx{\hat{\omega}^{b-}_{k+1}}{x} = 
    \begin{cases}
    l_{3,i-34} & \text{if } 34 < i \leq 37 \\
    \textbf{0}_{3,1} &\text{otherwise}.
    \end{cases}
 \]where 
 \begin{itemize}  
 \item $l_{x,y}$ is a zero vector of length $x$ with 1 at the $y^{th}$ position.
 \item $\textbf{0}_{m,n}$ is the zero matrix of dimensions $m$ and $n$.
 \end{itemize}

\item The Jacobian matrix corresponding to the joint angles $\hat{q}_{k+1}^-$ is
\begin{equation}
\label{eq:dq_msrdx}
\dfdx{\hat{q}_{k+1}^-}{x} = \left(\dfdx{\hat{q}_{k+1}^-}{x_{1}}, \dfdx{\hat{q}_{k+1}^-}{x_{2}}, \cdots , \dfdx{\hat{q}_{k+1}^-}{x_{62}}\right) \in \Re^{25 \times 62}
\end{equation}
 \[
 \dfdx{\hat{q}_{k+1}^-}{x_{i}} =
 \begin{cases}
 l_{25,i-6} & \text{if } 6 < i \leq 31 \\
 \textbf{0}_{25,1} & \text{otherwise}.
 \end{cases}
 \]

\item The Jacobian matrix corresponding to the angular velocity of joint angles  $\hat{\dot{q}}_{k+1}^-$ is 
\begin{equation}
 \label{eq:ddq_msrdx}
\dfdx{\hat{\dot{q}}_{k+1}^-}{x} = \left(\dfdx{\hat{\dot{q}}_{k+1}^-}{x_{1}}, \dfdx{\hat{\dot{q}}_{k+1}^-}{x_{2}}, \cdots , \dfdx{\hat{\dot{q}}_{k+1}^-}{x_{62}}\right) \in \Re^{25 \times 62}
\end{equation}
  \[
 \dfdx{\hat{\dot{q}}_{k+1}^-}{x_{i}} =
 \begin{cases}
 l_{25,i-37} & \text{if } 37 < i \leq 62 \\
 \textbf{0}_{25,1} & \text{otherwise}.
 \end{cases}
 \]

 \item The measurement equation for the contact points in right and left foot are 
 $$ \hat{s}_{cnt,k+1}=
 \begin{bmatrix}
 \hat{s}_{r,k+1}^- \\ \hat{s}_{l,k+1}^-
 \end{bmatrix} 
 = \begin{bmatrix}
 \hat{H}_{r,k+1}^- s  \\ \hat{H}_{l,k+1}^- s
   \end{bmatrix}
	$$ 
where $s$ can be zero or more points in the set $ \lbrace s_A,s_B,s_C,s_D \rbrace$. The points in contact $s$ is determined by the Algorithm \ref{alg:cnt_switch}. 

 The Jacobian matrix of the contact measurements is 
\begin{equation}
    \label{eq:dscnt_msrdx}
    \begin{split}
    \dfdx{\hat{s}_{cnt,k+1}^-}{x} &= \left[
    \begin{aligned}
    \dfdx{\hat{s}_{r,k+1}^-}{x}  \\ \dfdx{\hat{s}_{l,k+1}^-}{x}
    \end{aligned} \right] = \left[
    \begin{aligned}
    \dfdx{\hat{H}_{r,k+1}^-}{x}s \\ \dfdx{\hat{H}_{l,k+1}^-}{x}s 
    \end{aligned}\right]
 \\ \vspace{5mm} \\
     \dfdx{\hat{H}_{f,k+1}^-}{x} = &\left( \dfdx{\hat{H}_{f,k+1}^-}{x_1}, \dfdx{\hat{H}_{f,k+1}^-}{x_2},\cdots, \dfdx{\hat{H}_{f,k+1}^-}{x_{62}} \right) \hspace{5mm} f \in \{r,l\}
     \end{split}
\end{equation}
where $\dfdx{\hat{H}_{f,k+1}^-}{x}$ is the partial derivative of homogeneous transformation matrix of a foot with respect to the system states [Appendix \ref{sec:htm}].
\end{enumerate}
The measurement sensitivity matrix $\hat{C}_{k+1}^-$ of the system is given by Equations \ref{eq:dacc_msrdx}, \ref{eq:dw_msrdx}, \ref{eq:dq_msrdx}, \ref{eq:ddq_msrdx}and \ref{eq:dscnt_msrdx}:
\begin{equation}
    \label{eq:msr_mat}
    \hat{C}^-_{k+1} = \left(
   \begin{aligned}
   \dfdx{\hat{acc}_{k+1}^{b-}}{x} \\
   \dfdx{\hat{\omega}^{b-}_{k+1}}{x} \\
    \dfdx{\hat{q}_{k+1}^-}{x} \\
    \dfdx{\hat{\dot{q}}_{k+1}^-}{x} \\
    \dfdx{\hat{s}_{cnt,k+1}^-}{x} \\
   \end{aligned}
	 \right) \in \Re^{80 \times 62}.
\end{equation}

\begin{comment}
\paragraph{Observability:}
State space representation of a linear system is,
\begin{equation}
\label{eq:dyn_l}
\begin{split}
\dot{x} &= Ax + Bu\\
y &= Cx + Du.
\end{split}
\end{equation}
where, $x \in \Re^{n}$ is the vector representing the states of the system. $u \in \Re^{p}$ is the vector of inputs, $y \in \Re^{m}$ is the vector of outputs of the system. $A \in \Re^{n \times n}$ is the system matrix. $B \in \Re^{n \times p}$ is the matrix relating state and input, $C \in \Re^{m \times n}$ is the measurement matrix relating output and state, $D \in \Re^{m \times p}$ is the matrix relating input and output of the system.

Linearising a nonlinear system in Equation \ref{eqn:nl_sys} at some operating point will lead to linear system of form Eq. \ref{eq:dyn_l}. For a linear system to be observable, it should satisfy
\begin{equation}
obs =
\begin{pmatrix}
C\\ CA \\ CA^{2}\\ \vdots \\ CA^{n-1}
\end{pmatrix}
, rank(obs) =n
\end{equation}
For our system to be observable $rank(obs) = 62$.
\end{comment}
%\textbf{ Make plots from files act=datsrc/ROBOT-TILT-0807.mat est=estimates-data/est-090701.mat or *080703.mat }
%\end{document}


% Experiments 
\section{Experiment}
The estimation problem is tested in Matlab Simulink. The experimental setup consists of the Open HRP3 simulator and the state estimator. The OpenHRP3 (Open Architecture Human-centered Robotics Platform version 3) is an integrated software platform for robot simulations and software developments \footnote{\url{http://www.openrtp.jp/openhrp3/en/about.html}}. It was developed as a cooperative work of University of Tokyo, General Robotix. Inc and National Institute of Advanced Industrial Science and Technology(AIST).

%The Kane Screw algortihm is used as the rigid body 

\begin{figure}
    % We need layers to draw the block diagram
\pgfdeclarelayer{background}
\pgfdeclarelayer{foreground}
\pgfsetlayers{background,main,foreground}

% Define a few styles and constants
\tikzstyle{sensor}=[draw, fill=blue!20, text width=5em,text centered, minimum height=2.5em]
\tikzstyle{system} = [sensor, text width=6em, fill=green!30, 
    minimum height=12em, rounded corners]
\tikzstyle{input} = [coordinate]
\tikzstyle{sum} = [draw, fill=blue!20, circle, node distance=1cm]
%\tikzstyle{output} = [coordinate]
\def\blockdist{0.5}
\def\edgedist{0.75}
\begin{tikzpicture}
	% Define the nodes in the picture
	\node (sys_in)[yshift=1cm]{$x_0$};
	\node (u_in) [below of=sys_in,node distance=1cm]{$u$};
	\node (sys_u)[input,right of=u_in,node distance=1cm]{};
    \node (sim_sys) [system,right of=sys_in,node distance=3cm] {Double pendulum system};
    \node (sys_noise)[above of=sim_sys,node distance=3cm]{$n_w$};
    \node (msr_noise)[right of=sys_noise,node distance=2cm]{$n_v$};
    \node (msr_add)[sum,right of=sim_sys,node distance=2cm]{};
    \node (estimator) [system,right of=msr_add,node distance =3cm]{ Estimator};
    \node (est_in) [right of=sim_sys,node distance=3cm,yshift=1.5cm]{$\hat{x}_0$};
    \node (est_u) [input,right of =sim_sys, node distance=2cm,yshift=-3cm]{foo};
    \node (est_out)[right of=estimator,node distance=3cm]{$\hat{x}$};
    
    % Define the edges in the picture
    \draw [->] (sys_in) --node{}(sim_sys.west);
    \draw [-] (u_in) --node{}(sys_u);
    \draw [->] (sys_u) --node{}+(\edgedist,0);
    \draw [-] (sys_u) |-node{}(est_u);
    \draw [->] (est_u) |-node[pos=0.7,above]{$u$}(estimator.-130);
    \draw [->] (est_in) --node{}+(\edgedist,0);
    \draw [->] (sys_noise) --node{}(sim_sys.north);
    \draw [->] (msr_noise) --node{}(msr_add.north);
    \draw [->] (sim_sys.east) --node[above]{}(msr_add.west);
    \draw [->] (msr_add.east) --node[above]{$y$}(estimator.west);
    \draw [->] (estimator) --node{}(est_out);
\end{tikzpicture}
    \caption{Experimental setup of TORO}
\end{figure}

\textbf{explanation}



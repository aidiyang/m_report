\section{Inverted Double Pendulum model} 
A preliminary version of planar robot model resembles the Inverted Double pendulum. Figure \ref{fig:idp} shows robot standing with one foot on the ground, the first link represents the robot's foot, second link is attached to the first link through a joint which is the ankle of the robot. The upper mechanical structure is approximated as a single mass for the case of simplicity. Torque $\tau$ is applied to ankle joint. 
\begin{figure}[h]
	\centering
	\includegraphics[scale=0.75]{Bilder/doublePendulum1.png}
	\caption{Inverted Double Pendulum}	
	\label{fig:idp}
\end{figure}
\begin{equation}
	 q = 
	\begin{pmatrix}
		q_{1}\\
		q_{2}
	\end{pmatrix}
	 \dot{q} = 
	\begin{pmatrix}
		\dot{q_{1}}\\
		\dot{q_{2}}
	\end{pmatrix}
\end{equation}

The equation of motion of the inverted double pendulum is derived by Lagrange formulation, the generalized coordinates of the system are $q_1,q_2$ which represents the angles of the joints, $\dot{q_1},\dot{q_2}$ represent the velocities of corresponding joints as shown in figure \ref{fig:idp}.
\begin{equation}
	M(q)_{2\times2}.
	\begin{pmatrix}
		\ddot{q_{1}} \\
		\ddot{q_{2}} 
	\end{pmatrix}
	+ C(q,\dot{q})_{2\times2}.
	\begin{pmatrix}
		\dot{q_{1}} \\
		\dot{q_{2}} 
	\end{pmatrix}
	+ g(q)_{2\times 1} = \tau_{2\times 1}
\end{equation}

\begin{equation}
	\label{eq:dyn_eq}
	\begin{pmatrix}
		\ddot{q_{1}} \\
		\ddot{q_{2}} 
		\end{pmatrix}
	= M(q)^{-1} \left( -C(q,\dot{q}).\dot{q} - g(q) + \tau \right )
\end{equation}

To convert the equations into ODE's, assume 
$$ x_1 = q_1, x_2 = \dot{q_1}, x_3 = q_2, x_4 = \dot{q_2}$$
substitute the above equations into \eqref{eq:dyn_eq}. The resulting non linear dynamic equation is of form

\begin{equation}
\begin{split}\label{eq:dyn_ode}
	\dot{x}  = f(x,u),	\\
	x = 
	\begin{pmatrix}
		x_1 \\
		x_2 \\
		x_3 \\
		x_4
	\end{pmatrix}
\end{split}
\end{equation}
\eqref{eq:dyn_ode} is the model prediction equations in Kalman filter. The measurement equation is,

\begin{equation}
	y= 
	\begin{pmatrix}
		q_2\\
		\dot{q_2}\\
		a_{py}\\
		a_{pz}\\
		\omega_x\\
		Fb_y\\
		Fb_z \\
	\end{pmatrix}
\end{equation}

where,
\begin{itemize}
\item
 $q_2, \dot{q_2}$- angle and velocity of ankle joint measured by encoders
\item 
$a_{py},a_{pz}$- cartesian accelerations along \emph{x and z axis} measured by IMU(Inertial Measurement Unit)
\item
$\omega_x$ - angular acceleration around \emph{x axis} measured by gyroscope
\item
$Fb_y,Fb_z$- ground reactional froces action along \emph{ y and z axis} measured by FTS(Force Torque Sensor)
\end{itemize}
The estimated states are $$q_1, \dot{q_1}$$ are the underactuated degrees of freedom of the system.

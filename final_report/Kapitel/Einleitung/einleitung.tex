\chapter{Introduction}
\label{sec:einleitung}

\section{Motivation}
The field of \emph{Robotics} have seen a tremendous development since the introduction of the term by \emph{Isaac Asimov} in 1940s. The fundamental components of robotic systems are mechanical structure, actuators, sensors and controller. Robotic system ranges from simple \emph{Cartesian manipulator} to the complex \emph{Humanoids}. \emph{Industrial robots} are robots that are used in applications such as palletizing, material loading and unloading, part sorting, packaging etc. These robots usually operate in the structured environment whose geometrical or physical characteristics are known in priori. They are pre programmed to execute the set of tasks. These robots have largely aided the automation of manufacturing processes in the industries. \emph{Mobile robots} that are used in the environments where human beings can hardly survive or be exposed to unsustainable risks are called \emph{Field robots}. \emph{Field robots} normally operate in the unstructured environments, where the geometry or physical characteristics are not know in priori. Mars rover \emph{Curiosity} is one such example. Locomotion in these robots are achieved either by wheels or by mechanical legs. Operating in the unknown environments and dynamic balancing of mechanical structure demands advanced control schemes for \emph{Field robots}.
\begin{comment}
a) structural improvement is limited
b) actuator is also limited
c) need for dexterous manipulation
c) Evolution in control strategy.
e) Model based controls.
d) limited sensors available.
f) need for filters
Chapter 2 : Discuss about walking 
Chapter 3 : State Estimation and Kalman Filtering
Chapter 4 : Results
 Even \emph{mobile robots} had some serious limitations in their deployability. For normal functioning these robots needed a flat surface which would be suitable for their smooth navigation and they do not have flexibility like human beings. For example, navigating through stairs is not possible with \emph{mobile robots}. These limitations ultimately encouraged the idea of having robots which have similar structural properties like humans, as it could be easily deployed in the human environment. This lead to the idea of Humanoid robots. Since then, research in this field have gained its importance. \emph{Toro} is the humanoid developed by DLR shown in Figure \ref{fig:toro}. \emph{Toro} is first built as a bipedal walker, which could execute human-like walking gait.
\end{comment}

\begin{figure}[t]
  \centering
  \includegraphics[width=0.5\textwidth]{Bilder/Abbildung.pdf}
  %\caption{Eine Beispiel-Abbildung \cite{testref}.}
  \caption{Eine Beispiel-Abbildung.}
  \label{fig:bsp}
\end{figure}


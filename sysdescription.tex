\documentclass[12pt]{article}
% Kalman Filtering forumulation
\usepackage{amsmath}
\usepackage{verbatim}
\usepackage{fixltx2e}

\def\dfdx#1#2{\frac{\partial {#1}}{\partial {#2}}}
\begin{document}
\newpage
\section{System Description}
\emph{Toro} is modelled as floating base dynamic model.
\begin{equation}
	\label{eq:motion}
	\begin{split}
	\ddot{q} = M(q)^{-1}(&-C(q,\dot{q})\dot{q} - g(q) + (J_{bl}^{b})^{T}F_{l}^{b} +(J_{br}^{b})^{T}F_{r}^{b} + \tau) \\
	\text{where,}\\
	q_{base} &= (p_{f}^{T},\theta_{f}^{T})^{T} \in \Re^{6} \\
	q_{j} &= (q_{1},q_{2},\ldots , q_{25})^{T} \in \Re^{25}\\
	q &= (q_{f}^{T}, q_{j}^{T})^{T} \in \Re^{31}\\
	\end{split}
\end{equation}
\emph{q} is the generalized coordinates. The coordinates of the floating base $q_{f}$ are defined with reference to spatial frame, whereas $q_{j}$ are defined with reference to body frame. $q_{f} = (p_{f}^{T},\theta_{f}^{T})^{T}$, where $p_{f}=(x,y,z)^{T}$ is a vector representing  the position of origin of floating base in spatial frame and $\theta_{f}=(\theta_{x},\theta_{y},\theta_{z})^{T}$ is a vector of Euler angles that describes the rotation of base frame with respect to spatial frame.$q_{j} \in \Re^{25}$ is the vector of joint angles. $\dot{q} =  (\hat{V}_{f}^{b},\dot{q_{j}})^{T} \in \Re^{31}$ is the vector of generalized velocities. $\hat{V}_{f}^{b} \in \Re^{6}$ is the velocity of floating base in body frame. $\dot{q}_{j} \in \Re^{25}$ is the vector of joint velocities. $\ddot{q}\in \Re^{31}$ is the vector of generalized accelerations. $M(q)\in \Re^{31 \times 31}$ is the inertia matrix, $C(q,\dot{q})\in \Re^{31 \times 31}$ is the matrix accounting for centrifugal and Coriolis forces. $g(q) \in \Re^{31}$ is the gravity vector. $\tau \in \Re^{31}$ is the vector of actuating torques acting on $q_{j}$, where the first six components are zero because those degrees of freedom corresponding to $q_{f}$ are not actuated.$(J_{br}^{fb})^{T},(J_{bl}^{fb})^{T} \in \Re^{31 \times 6}$ are the Jacobian matrices transforming the wrenches $F_{r}^{fb},F_{l}^{fb} \in \Re^{6}$ applied in the right and left foot to joint torques.
\paragraph{State Space representation:}
General state space representation of a non linear system
\begin{equation}
\label{eq:dyn_nl}
	\begin{split}
	\dot{x} = f(x,u)\\
	y = h(x,u)
	\end{split}
\end{equation}
where, $x \in \Re^{n}$ is the vector representing the states of the system. $u \in \Re^{p}$ is the vector of inputs acting on the system. $y \in \Re^{m}$ is the vector of outputs of the system.\\
State space representation of \emph{Toro}
\begin{equation}
\label{eq:toro}
	\begin{split}
	\dot{x} = &
	\begin{pmatrix}
	T_{A}(\theta_{f})^{-1}Ad_{sb}\hat{V}_{f}^{b}\\
	\dot{q}_{j}\\
	M(q)^{-1}(-C(q,\dot{q})\dot{q} -g(q) + (J_{bl}^{b})^{T}F_{l}^{b} +(J_{br}^{b})^{T}F_{r}^{b} + \tau)	
	\end{pmatrix}
	\\&	y = 
	\begin{pmatrix}
	q_{j} \\ \dot{q}_{j} \\ \ddot{p}_{f} \\ \dot{\theta}_{f}^{s}\\ p_{c}^{b}\\ \hat{V}_{c}^{b}
	\end{pmatrix}
	\\ \text{where, }\\ x &= (q^{T},\dot{q}^{T})^{T} \in \Re^{62}\\
		T_{A}(\theta_{f}) &=
	\begin{pmatrix}
	\textbf{I} &\textbf{0} \\
	\textbf{0} &T(\theta_{f})
	\end{pmatrix} 
	\\T(\theta_{f}) &=
	\begin{pmatrix}
	1 &0 &\sin(\theta_{y})\\
	0 &\cos(\theta_{x}) &-\sin(\theta_{x})\cos(\theta_{y})\\
	0 &\sin(\theta_{x}) &\cos(\theta_{x})\cos(\theta_{y})\\
	\end{pmatrix}	\\
	\ddot{p}_{f}^{b} &= \ddot{q}(1,2,3)^{T} + R_{sb}^{T} 
	\begin{pmatrix}
	0 \\ 0 \\ 9.81
	\end{pmatrix}\\
	\dot{\theta}_{f}^{b} &= R_{sb}^{T}T(\theta_{f})^{-1}R_{sb}\omega_{f}^{b}\\
	p_{c}^{b} &= (p_{a,r}^{b}, p_{b,r}^{b}, p_{a,l}^{b}, p_{b,l}^{b} )^{T} \\
	 \text{ i.e } &p_{a,r}^{b} = H_{sr}^{-1} p_{a,r}^{s},p_{b,r}^{b} = H_{sr}^{-1} p_{b,r}^{s},p_{a,l}^{b} = H_{sl}^{-1} p_{a,l}^{s},p_{b,l}^{b} = H_{sb}^{-1} p_{b,l}^{s}  \\
	\hat{V}_{c}^{b} &= (\hat{V}_{r}^{b},\hat{V}_{l}^{b})^{T} \text{ i.e } \hat{V}_{r}^{b} = J_{r}\dot{q} \text{ and } \hat{V}_{l}^{b} = J_{l}\dot{q}
	\end{split}
	\end{equation}
\begin{itemize}
\item $T(\theta_{f})$ is the matrix that transforms the angular velocity $\omega_{f}^{s}$ to the time derivative of Euler angles $\dot{\theta}_{f}^{s}$. i.e $\omega_{f}^{s}=T(\theta_{f}) \dot{\theta}_{f}^{s}$. 
\item $Ad_{sb} \in \Re^{6 \times 6}$ is the adjoint transformation matrix that transforms the body velocity to spatial velocity. 
\item $\ddot{p}_{f}^{b}$ is the vector of Cartesian accelerations of the floating base in body frame (measured by IMU).	$\ddot{q}(1,2,3)$ is the first three elements of $\ddot{q}$ computed by Eq. \ref{eq:motion}. $R_{sb}^{T}(0,0,9.81)^{T}$ is the term added to compensate for the gravity measured by IMU. 
\item $\dot{\theta}_{f}^{b} $ is the vector of angular rates of floating base in body frame (measured by Gyroscope). 
\item $p_{c}^{b}$ is the vector of position constraints of right and left foot. These positions constraints are the static points on the foot of the robot. $H_{sx}$ is the homogeneous transformation matrix from frame \emph{x} to spatial frame. $p_{a,r}^{s}, p_{b,r}^{s}, p_{a,l}^{s}, p_{b,l}^{s}$ are known points which are constant with respect to spatial frame.
\item $\hat{V}_{c}^{b}$ are velocity constraints on right and left foot.
\end{itemize}
%Tori = [1 0 sin(p(5)); 0 cos(p(4)) -cos(p(5))*sin(p(4)); 0 sin(p(4)) cos(p(5))*cos(p(4))];
\paragraph{Observability:}
State space representation of a linear system is,
\begin{equation}
\label{eq:dyn_l}
\begin{split}
\dot{x} &= Ax + Bu\\
y &= Cx + Du.
\end{split}
\end{equation}
where, $x \in \Re^{n}$ is the vector representing the states of the system. $u \in \Re^{p}$ is the vector of inputs, $y \in \Re^{m}$ is the vector of outputs of the system. $A \in \Re^{n \times n}$ is the system matrix. $B \in \Re^{n \times p}$ is the matrix relating state and input, $C \in \Re^{m \times n}$ is the measurement matrix relating output and state, $D \in \Re^{m \times p}$ is the matrix relating input and output of the system.

Linearising a non linear system in Eq. \ref{eq:dyn_nl} at some operating point will lead to linear system of form Eq. \ref{eq:dyn_l}. For a linear system to be observable, it should satisfy
\begin{equation}
obs =
\begin{pmatrix}
C\\ CA \\ CA^{2}\\ \vdots \\ CA^{n-1}
\end{pmatrix}
, rank(obs) =n
\end{equation}
Let $g_{sb} = T_{A}(\theta_{f})^{-1}Ad_{sb}$, then from Eq. \ref{eq:toro},
$$\alpha = g_{sb} \hat{V}_{f}^{b}$$
\begin{equation}
\label{eq:dtoro_alpha}
\dfdx{\alpha}{x} = \left(\dfdx{\alpha}{x_{1}}, \dfdx{\alpha}{x_{2}}, \cdots , \dfdx{\alpha}{x_{62}}\right) \in \Re^{6 \times 62}
\end{equation}
\[
 \dfdx{\alpha}{x_{i}} = 
  \begin{cases}
   g_{sb}adj_{J_{f}^{i}} \hat{V}_{f}^{b} & \text{if } i \leq 3 \\
  -T(\theta)^{-1}\dfdx{T}{x_{i}}\alpha + g_{sb} adj_{J_{f}^{i}} \hat{V}_{f}^{b} & \text{if } i > 3 \text{ or } i \leq 6 \\
   g_{sb} adj_{J_{f}^{i}} \hat{V}_{f}^{b} & \text{if } i > 6 \text{ or } i \leq 31 \\
   col(g_{sb},i-31) & \text{if } i > 31 \text{ or } i \leq 37 \\
  \end{cases}
\]
where,
\begin{itemize}
\item $col(X,i)$ - represents the column of matrix $X$.
\item $adj_{J_{f}^{i}}$ - Skew symmetric matrix formed by $col(J_{f},i)$
\item $\dfdx{Ad_{sb}}{x} = Ad_{sb} adj_{J_{f}^{i}} $ (Gianluca)
\end{itemize}
\begin{equation}
\label{eq:dtoro_dq}
\dfdx{\dot{q}}{x} = \left(\dfdx{\dot{q}}{x_{1}}, \dfdx{\dot{q}}{x_{2}}, \cdots , \dfdx{\dot{q}}{x_{62}}\right) \in \Re^{25 \times 62}
\end{equation}
\[
\dfdx{\dot{q}}{x_{i}} = 
	\begin{cases}
	1 & \text{if } i > 37 \\
	0 & \text{if } i \le 37  \\
	\end{cases}
\]
Let $B = -C(q,\dot{q})\dot{q} - g(q) + J_{l}^{T}F_{l} + J_{r}^{T}F_{r} + \tau$, then from Eq. \ref{eq:toro},
$$\Lambda = M(q)^{-1}B$$
 \begin{equation}
 \label{eq:dtoro_lambda}
\dfdx{\Lambda}{x} = \left(\dfdx{\Lambda}{x_{1}}, \dfdx{\Lambda}{x_{2}}, \cdots , \dfdx{\Lambda}{x_{62}}\right) \in \Re^{31 \times 62}
\end{equation}
where,
\[
\dfdx{\Lambda}{x_{i}} = 
\left\{ 
\!\begin{aligned}
	& \left. \!\begin{aligned}
	-M^{-1}\dfdx{M}{x_{i}}\Lambda + &M^{-1}\left(-\dfdx{C}{x_{i}}\dot{q} -\dfdx{g}{x_{i}}        + \left(\dfdx{J_{r}^{b}}{x_{i}}\right)^{T}F_{r}\right) \\
	& + M^{-1}\left(\dfdx{J_{l}^{b}}{x_{i}}\right)^{T}F_{l}
	\end{aligned} \right\}& \text{if } i \leq 31 \\
&-M^{-1}\dfdx{M}{x_{i}}\Lambda + M^{-1}\left(-\dfdx{C}{x_{i}}\dot{q}- col(C,i-31)\right) & \text{if } i > 31  \\
\end{aligned}
\right.
\]
System matrix is given by Eq. \ref{eq:dtoro_alpha},\ref{eq:dtoro_dq} and \ref{eq:dtoro_lambda}
\begin{equation}
\label{eq:sys_A}
A = \left(
\begin{aligned}
\dfdx{\alpha}{x} \\
\dfdx{\dot{q}}{x} \\
\dfdx{\lambda}{x}
\end{aligned} \right)
\in \Re^{62 \times 62}
\end{equation}
For computation of measurement matrix \emph{C} the derivative of y in Eq. \ref{eq:toro} with respect to system state is computed.
\begin{equation}
\label{eq:dmsr_q}
\dfdx{q}{x} = \left(\dfdx{q}{x_{1}}, \dfdx{q}{x_{2}}, \cdots , \dfdx{q}{x_{62}}\right) \in \Re^{25 \times 62}
\end{equation}
 \[
 \dfdx{q}{x_{i}} =
 \begin{cases}
 1 & \text{if } 7 \leq i \leq 31 \\
 0 & \text{otherwise}
 \end{cases}
 \]
 \begin{equation}
 \label{eq:dmsr_dq}
\dfdx{\dot{q}}{x} = \left(\dfdx{\dot{q}}{x_{1}}, \dfdx{\dot{q}}{x_{2}}, \cdots , \dfdx{\dot{q}}{x_{62}}\right) \in \Re^{25 \times 62}
\end{equation}
  \[
 \dfdx{\dot{q}}{x_{i}} =
 \begin{cases}
 1 & \text{if } 38 \leq i \leq 62 \\
 0 & \text{otherwise}
 \end{cases}
 \]
 \begin{equation}
 \label{eq:dmsr_dacc}
 \dfdx{\ddot{p}_{f}^{b}}{x} = \left( \dfdx{\ddot{p}_{f}^{b}}{x_{1}},\dfdx{\ddot{p}_{f}^{b}}{x_{2}}, \cdots , \dfdx{\ddot{p}_{f}^{b}}{x_{62}} \right) \in \Re^{3 \times 62}
 \end{equation}
 $$ \dfdx{\ddot{p}_{f}^{b}}{x_{i}} = \dfdx{\Lambda}{x}(1:3,:) + \left(\dfdx{R_{sb}}{x_{i}}\right)^{T}
 \begin{pmatrix}
 0 \\ 0 \\ 9.81
 \end{pmatrix}$$
where,\\
$\dfdx{\Lambda}{x}(1:3,:)$ represents the first three rows of $\dfdx{\Lambda}{x}$ in Eq. \ref{eq:dtoro_lambda}.
\begin{equation}
\label{eq:dmsr_dtheta}
\dfdx{\dot{\theta}_{f}^{b}}{x} = \left(\dfdx{\dot{\theta}_{f}^{b}}{x_{1}}, \dfdx{\dot{\theta}_{f}^{b}}{x_{2}}, \cdots , \dfdx{\dot{\theta}_{f}^{b}}{x_{62}} \right) \in \Re^{3 \times 62}
\end{equation}
\[
\dfdx{\dot{\theta}_{f}^{b}}{x_{i}} = 
\left\{ 
\!\begin{aligned}
&\left(\dfdx{R_{sb}}{x_{i}}\right)^{T}T^{-1}R_{sb} \omega_{f}^{b}+ R_{sb} ^{T}T^{-1} \left(\dfdx{R_{sb}}{x_{i}}\right)\omega_{f}^{b} & \text{if } 1 \leq i \leq 3 \\
&\!\begin{aligned}%[b]
	\left(\dfdx{R_{sb}}{x_{i}}\right)^{T}T^{-1} & R_{sb} \omega_{f}^{b} - R_{sb} ^{T}T^{-1} \left(\dfdx{T}{x_{i}}\right)T^{-1}R_{sb} \omega_{f}^{b} \\
	 & + R_{sb} ^{T}T^{-1} \left(\dfdx{R_{sb}}{x_{i}}\right)\omega_{f}^{b}
    \end{aligned}       & \text{if } 4 \leq i \leq 6 \\[1ex]
&\left(\dfdx{R_{sb}}{x_{i}}\right)^{T}T^{-1}R_{sb} \omega_{f}^{b}+ R_{sb} ^{T}T^{-1} \left(\dfdx{R_{sb}}{x_{i}}\right)\omega_{f}^{b} & \text{if } 7 \leq i \leq 31 \\
&col(R_{sb}^{T}T^{-1}R_{sb},i) & \text{if } 35 \leq i \leq 37 \\
&0 & \text{ otherwise} 
\end{aligned}
\right.
\]
\begin{equation*}
\left.\begin{aligned}
B'&=-\partial\times E,\\
E'&=\partial\times B - 4\pi j,
\end{aligned}
\right\}
\qquad \text{Maxwell's equations}
\end{equation*}
\begin{equation}
\label{eq:dmsr_dpc}
\dfdx{p_{c}^{b}}{x} = \left(\dfdx{p_{c}^{b}}{x_{1}}, \dfdx{p_{c}^{b}}{x_{2}}, \cdots , \dfdx{p_{c}^{b}}{x_{62}}\right) \in \Re^{12 \times 62}
\end{equation}
\[
\dfdx{p_{c}^{b}}{x_{i}} =
\begin{cases}
\left(
\begin{aligned}
-H_{sr}^{-1}\dfdx{H_{sr}}{x_{i}}H_{sr}^{-1}p_{a,r}^{s} \\
-H_{sr}^{-1}\dfdx{H_{sr}}{x_{i}}H_{sr}^{-1}p_{b,r}^{s} \\
-H_{sl}^{-1}\dfdx{H_{sl}}{x_{i}}H_{sl}^{-1}p_{a,l}^{s} \\
-H_{sl}^{-1}\dfdx{H_{sl}}{x_{i}}H_{sl}^{-1}p_{b,l}^{s}
\end{aligned} \right)
& \text{if } 1 \leq i \leq 31 \\
0 &\text{otherwise}
\end{cases}
\]
 \begin{equation}
 \label{eq:dmsr_dvc}
\dfdx{\hat{V}_{c}^{b}}{x} = \left(\dfdx{\hat{V}_{c}^{b}}{x_{1}}, \dfdx{\hat{V}_{c}^{b}}{x_{2}}, \cdots , \dfdx{\hat{V}_{c}^{b}}{x_{62}}\right) \in \Re^{12 \times 62}
\end{equation}
\[
\dfdx{\hat{V}_{c}^{b}}{x_{i}} = 
	\begin{cases}
	\left(
	\begin{aligned}
	\dfdx{J_{r}^{b}}{x_{i}}\dot{q} \\
	\dfdx{J_{l}^{b}}{x_{i}}\dot{q} \\
	\end{aligned} \right)
	& \text{if } 1 \leq i \leq 31 \\
	\begin{pmatrix}
	col(J_{r}^{b},i)\\ col(J_{l}^{b},i)
	\end{pmatrix}
	 	& \text{if } 32 \leq i \leq 62
	\end{cases}
\]
The measurement matrix of the system is given by Eq. \ref{eq:dmsr_q}, \ref{eq:dmsr_dq}, \ref{eq:dmsr_dacc}, \ref{eq:dmsr_dtheta}, \ref{eq:dmsr_dpc}, \ref{eq:dmsr_dvc}
\begin{equation}
C = \left(
   \begin{aligned}
   \dfdx{q}{x} \\
	 \dfdx{\dot{q}}{x}\\
	 \dfdx{\ddot{p}_{f}^{b}}{x}\\
	 \dfdx{\dot{\theta}_{f}^{b}}{x}\\
	 \dfdx{p_{c}^{b}}{x}\\
	 \dfdx{\hat{V}_{c}^{b}}{x} 
   \end{aligned}
	 \right) \in \Re^{80 \times 62}
\end{equation}

\begin{comment}
This is a code to format lengthy equations.
\[
  \text{left hand side} =
  \begin{cases}
    \!\begin{aligned}%[b]
       & \text{a very long expression} \\
       & + \text{that continues on the next line}
    \end{aligned}           & \text{1st condition} \\%[1ex]
    \text{short expression} & \text{2nd condition}
  \end{cases}
\]
\end{comment}
For our system to be observable $rank(obs) = 62$.
\end{document}
